% !TEX root = ../ck2.tex

\newpage
\section{Introduction}

This work is motivated by gauge field theory, particularly by \textit{colour--kinematics duality} and the \textit{double copy construction}.
From the algebro-homotopical perspective on perturbative field theory, surveyed for example in \cite{Hohm2017LInfinityAlgebrasGauge, Jurco2020PerturbativeQFTandHomotopyAlgebras, Jurco2019LInfinityBVReview}, a classical field theory is encoded as a cyclic \(\rL_\infty\)-algebra \(\big(\cL, \set{\ell_n}, \langle\cdot, \cdot \rangle \big)\)---a model for homotopy Lie algebras with a compatible non-degenerate inner product---whose Maurer--Cartan functional
\[
S(a) \defeq \sum_{n \geqslant 1} \frac{1}{(n+1)!}\,\big\langle a, \ell_n(a^{\otimes n})\big\rangle
\]
defines the action.
For many gauge theories of interest, including Chern--Simons and Yang--Mills, their \(\rL_\infty\)-algebra splits as a tensor product
\[
\cL \cong \cK \ot \g,
\]
where \(\g\) is a Lie algebra, the \textit{colour Lie algebra}, and \(\cK\) is a homotopy commutative algebra, the \textit{kinematic algebra}.
This \emph{colour-splitting} provides the algebraic framework to understand the duality between colour and kinematics, a phenomenon first observed by Bern--Carrasco--Johansson while studying amplitudes in Yang--Mills theory \cite{Bern2008NewRelationsGaugeTheory}.
There it was found that kinematic numerators obey the same antisymmetry and Jacobi relations as the colour factors, suggesting that the kinematic algebra \(\cK\) carries a Lie-type structure \cite{Monteiro2011KinematicAlgebraSelfDual}.

For instance, in Chern--Simons theory one has \(\cK = (\Omega(M), d, \wedge)\), the algebra of differential forms on an oriented 3-manifold.
Equipping \(M\) with a metric introduces the Hodge codifferential \(d^\star\), and the commutator \([d^\star, \wedge]\) endows \(\cK\) with a shifted Lie bracket giving rise to the kinematic Lie algebra of Chern--Simons theory \cite{BenShahar2022OffShellCKDualityCS}.

In this example, \(\cL\) is a dg Lie algebra, i.e., an \(\rL_\infty\)-algebra in which all operations beyond arity 2 vanish, reflecting the fact that Chern--Simons theory involves only cubic interactions.
The algebraic structure on the kinematic algebra \((\cK, d, \wedge, d^\star)\) is itself strict and motivates the notion of \emph{\(\cBV\)-algebra}, a weaker version of dg Batanin--Vilkovisky algebra \cite{Lian1993NewPerspectivesBRST, Getzler1994BVAlgebrasTQFT}, or \(\BV\)-algebra for short.
More precisely, a \(\cBV\)-algebra is a tuple \((A, d_A, m, \triangle)\) where \((A, d_A, m)\) is a dg commutative algebra and \(\triangle\) is a nilpotent operator of degree \(|\triangle| = -|d_A|\) and order at most 2.

\(\BV\)-algebras are special cases of \(\cBV\)-algebras, realized precisely when the \emph{obstruction} \([d_A, \triangle]\) vanishes.
In this case, the shifted Lie bracket \(\b = [\triangle, \m]\) makes \((A, d_A, \m, \b)\) into a dg Gerstenhaber algebra.
For differential forms, the obstruction \([d, d^\star]\) is precisely the failure of the de Rham differential and the Hodge codifferential to commute, yielding the Laplacian in Riemannian signature and the d’Alembertian in Lorentzian.

The concept of \(\cBV\)-algebra, or \textit{coexact \(\BV\)-algebra}, is dual---in a sense explained in \cref{ss:exact}---to the notion of \textit{exact \(\BV\)-algebra} arising in Poisson geometry \cite{DotsenkoShadrinVallette15, GuanMuro23}.
In the physics literature, \(\cBV\)-algebras are referred to as \(\BV^\Box\)-algebras, a term coined by M.~Reiterer in his influential preprint \cite{Reiterer2020HomotopyBVYMCK}.
These algebras have been employed to define colour--kinematics duality for theories with at most cubic interactions and to construct double copies of these \cite{Borsten2021DoubleCopyHomotopyAlgebras, Borsten2023KinematicLieAlgebrasTwistor, Borsten2023DoubleCopySDYM, Borsten2023TreeLevelCKPureSpinor, Bonezzi2024DoubleCopy3DCSKodairaSpencer, Borsten2025DoubleCopyFromTensorBV, BenShahar2025OffShellDoubleCopyBV}.
To move beyond the cubic case and incorporate higher-order interactions, as for instance up to quartic order in \cite{Bonezzi2022GaugeStructureDoubleField, Bonezzi2023GaugeInvariantDoubleCopyQuartic, Bonezzi2024WeaklyConstrainedDoubleField, Bonezzi2023GaugeIndependentKinematicSDYM}, M.~Reiterer also pioneered the use of a homotopy version of \(\BV^\Box\)-algebras, which he introduced through an ingenious albeit ad hoc definition.
The goal of the present paper is to provide a conceptual definition of homotopy \(\cBV\)-algebras and a concrete model for them, which we ground in the operadic calculus of \cite{LodayVallette12}.
As reviewed in \cref{ss:tools}, this enables the systematic application of key homotopical techniques to the resulting \(\cBV_{\!\infty}\)-algebras, including their homotopy transfer, rectification, \(\infty\)-morphisms, and deformation theory.

To facilitate the use of \(\cBV_{\!\infty}\)-algebras and the above techniques in physics, we give explicit descriptions of the generating operations and relations defining  \(\cBV_{\!\infty}\)-algebras, relate them to \(\rC_\infty\)- and \(\BV_\infty\)-algebras, and show explicitly how the quartic-level structure constructed in \cite{Bonezzi2023GaugeInvariantDoubleCopyQuartic, Bonezzi2024WeaklyConstrainedDoubleField} on the kinematic algebra of Yang--Mills theory fits naturally into our framework.
%In fact, as explained below, any extension of a kinematic algebra from a homotopy commutative algebra to a homotopy \cBV-algebra does as well.

\medskip

Conceptually, it is useful to situate \(\cBV_{\!\infty}\)-algebras in a more abstract context.
Within the relevant model category \cite{Hinich97}, the operad \(\cBV_{\!\infty}\) cofibrantly resolves the operad \(\cBV\), with this resolution fitting in the following commutative diagram, where the horizontal compositions are the canonical inclusions:
\begin{equation}\label{eq:diagram_main}
	\begin{tikzcd}[row sep=large, column sep=large]
		\Com_\infty \dar[->>, "\sim"] \arrow[r, >->, "\sim"] &
		\cBV_\infty \dar[->>, "\sim"] \arrow[r,->>] &
		\BV_{\!\infty} \dar[->>, "\sim"] \\
		\Com \arrow[r, "\sim"] &
		\cBV \arrow[r,->>] &
		\BV
	\end{tikzcd}
\end{equation}
Here \(\rightarrowtail\) denotes a cofibration, \(\twoheadrightarrow\) a fibration, and \(\xra{\sim}\) a weak equivalence.
The operads
\begin{equation*}\label{eq:cofibrant_operads_intro}
	\Com_\infty \defeq \Cobar\Com^\ac,
	\qquad
	\BV_{\!\infty} \defeq \Cobar\BV^\ac,
	\qquad
	\cBV_{\!\infty} \defeq \Cobar\cBV^\ac,
\end{equation*}
arise from applying the cobar construction to the Koszul dual cooperad of the operad governing the corresponding strict algebras.

\medskip\noindent
Two model-independent consequences of Diagram \eqref{eq:diagram_main} are:
\begin{enumerate}
	\item Any homotopy commutative algebra extends, up to homotopy, to a homotopy \(\cBV\)-algebra.
	\item A homotopy \(\cBV\)-algebra descends, up to homotopy, to a homotopy \(\BV\)-algebra if and only if its restriction to the homotopy fibre of the inclusion of \(\Com\) into \(\BV\) is nullhomotopic.
\end{enumerate}
In addition we have the following model-dependent algebraic counterparts:
\begin{enumerate}
	\item[(a1)] Any \(\rC_\infty\)-algebra defines a \(\cBV_{\!\infty}\)-algebra after arbitrarily choosing certain \textit{extra generating maps} described in \cref{ss:generating_maps}.
	\item[(a2)] A \(\cBV_{\!\infty}\)-algebra defines a \(\BV_{\!\infty}\)-algebra if and only if certain explicitly defined \textit{obstruction maps}, described in \cref{ss:obstruction_maps}, vanish.
\end{enumerate}
Moreover, the sets of extra generating maps and of obstruction maps are in a canonical bijection.

The picture that emerges in practice is that of a controlled extension process.
The kinematic data, presented as a \(\rC_\infty\)-algebra may always be promoted to a \(\cBV_{\!\infty}\)-algebra, but the promotion is not necessarily canonical: it depends on a choice of extra generating maps.
Each such new operation gives rise to an obstruction map, all of which vanish precisely when the resulting structure is that of a \(\BV_{\!\infty}\)-algebra.
In applications, the additional generating maps are tuned so as to control the obstructions in a physically meaningful way.

\medskip

An important use of the duality between colour and kinematics, which was pioneered in \cite{Bern2010PerturbativeQGDoubleCopy} at the scattering amplitude level, is the \textit{double copy construction}, in which the colour Lie algebra is formally replaced by the kinematic Lie algebra.
At cubic order, this construction has been understood off-shell using the tensor product of \(\cBV\)-algebras \cite{Borsten2025DoubleCopyFromTensorBV}.
For this tensor product to exist---like in the associative case and unlike the Lie case---it is necessary and sufficient that the operad  \(\cBV\) admits a \emph{diagonal}, a morphism of operads
\[
\Delta \colon \cBV \to \cBV \ot_H \cBV,
\]
where \(\ot_H\) denotes the Hadamard (or aritywise) tensor product of operads.
We describe a canonical diagonal for \(\cBV\) in \cref{ss:hopf} recovering the tensor product defined in \cite{Borsten2025DoubleCopyFromTensorBV}.

Since \(\cBV_{\!\infty}\) is a cofibrant resolution of \(\cBV\), the lifting property in the model category of operads ensures the existence of a diagonal for \(\cBV_{\!\infty}\):
\[
\begin{tikzcd}
	&[-5pt] &[-10pt] \cBV_\infty \otimes_H \cBV_\infty \hspace*{3pt} \ar[d, ->> ,"\sim", shift right=5pt]\\
	\cBV_\infty \ar[r, ->>, "\sim \ "] \arrow[urr, dashed, out=45, in=180] & \cBV \ar[r, "\Delta"] & \cBV \otimes_H \cBV.
\end{tikzcd}
\]
Consequently, one obtains the existence of a tensor product for \(\cBV_{\!\infty}\)-algebras, which in principle allows for a formulation of the double copy extending beyond cubic interactions.
The construction, however, depends on a specific choice of diagonal, a task that falls outside the scope of this work.

\medskip

Mathematically, the central technical contribution of this work is the extension of Koszul duality to quadratic-linear presentations with non-trivial differentials.
This refinement allows the operad \(\cBV\) to be treated within the established framework of operadic calculus and, in particular, enables the explicit construction of its Koszul replacement.
As explained in \cref{ss:reiterer}, the resulting notion of \(\cBV_{\!\infty}\)-algebras, a model of homotopy \(\cBV\)-algebras, is more general and better behaved than the \(\BV^\Box_\infty\)-algebras introduced in \cite{Reiterer2020HomotopyBVYMCK}, which we are unsure if defines a model for homotopy .

\subsection*{Outline}

In \cref{sec:algebras} we work at the level of algebras, avoiding the operadic constructions underlying our definitions.
We describe \(\cBV_{\!\infty}\)-algebras in terms of generating maps and relations, and clarify their relationships with \(\rC_\infty\)- and \(\BV_{\!\infty}\)-algebras.
We then introduce a weight filtration that organizes the hierarchy of homotopies and yields stricter forms of \(\cBV_{\!\infty}\)-algebras.
Finally, we illustrate this algebraic structure using the kinematic algebra of Yang–Mills theory as developed in \cite{Bonezzi2023GaugeInvariantDoubleCopyQuartic,Bonezzi2024WeaklyConstrainedDoubleField}.

\cref{sec:operads} constructs Diagram~\eqref{eq:diagram_main} and develops the homotopical theory of algebras over the operad \(\cBV_{\!\infty}\).
We begin in \cref{ss:strict_operads} by reviewing the operads \(\Com\), \(\BV\), and \(\cBV\) governing the strict algebraic structures of interest.
In \cref{ss:exact}, an independent subsection, we compare \(\cBV\) with the operad \(\eBV\) of exact \(\BV\)-algebras.
Section~\cref{ss:koszul_duals} extends Koszul duality to differential graded inhomogeneous presentations, and \cref{ss:cbv} applies this framework to \(\cBV\) to obtain an explicit model for its Koszul replacement \(\cBV_\infty\).
In \cref{ss:hopf} we show that tensor products of \(\cBV_{\!\infty}\)-algebras are well defined up to homotopy.
Section~\cref{ss:homotopy_fibre} studies the model-categorical relationship between \(\Com_\infty\), \(\cBV_\infty\), and \(\BV_\infty\).
In \cref{ss:homotopy_algebras} we prove that algebras over \(\cBV_{\!\infty}\) agree with the explicitly presented \(\cBV_{\!\infty}\)-algebras of \cref{sec:algebras}.
Finally, \cref{ss:tools} describes the deformation, obstruction, and homotopy theories of \(\cBV_{\!\infty}\)-algebras from the operadic perspective.

\nocite{Lada1993IntroSHLie, Costello2011RenormalizationEFT, CostelloGwilliam2021FactorizationAlgebrasQFTv2}

\subsection*{Outlook}

Our focus in this paper is on the non-cyclic aspects of homotopy \(\cBV\)-algebras.
By contrast, the \(\rL_\infty\)-algebras arising in field theory typically carry a cyclic structure.
Accordingly, in examples of colour-splitting the associated kinematic \(\rC_\infty\)-structures are also cyclic.
Extending the present framework to cyclic \(\cBV_{\!\infty}\)-algebras, as well as to their quantum counterparts, constitutes an important direction for future research.