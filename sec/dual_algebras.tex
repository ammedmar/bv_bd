% !TEX root = ../ck2.tex

\section{Koszul dual Batalin--Vilkovisky algebras are Beilinson--Drinfeld algebras}

In this section, we make explicit the notions of algebras over the Koszul dual operads. 
The first case is classical: \(\Com^!\)-algebras are Lie algebras, see \cite[Section~13.1.5]{LodayVallette12}.

\subsection{Koszul dual \(\BV\)-algebras are \(\mathrm{BD}\)-algebras}
For the first time, we make fully explicit the notion of a dg \(\BV^!\)-algebra.

\begin{proposition}\label{prop:DualBValg}
	The desuspension of a differential graded \(\BV^!\)-algebra structure on \(sA\) is 
	a Gerstenhaber \(\mathbb{K}[\hbar]\)-algebra \((A, m ,\{\, , \,\} )\) on a chain 
	\(\mathbb{K}[\hbar]\)-complex  \((A,d)\), with \(|\hbar|=-2\), such that the differential \(d\) is a derivation 
	with respect to \(\{\, , \,\}\) and such that 
	\begin{equation}\label{eq:dKoszuldual}
	d(m(a,b)) = m(d(a),b) + (-1)^{|a|} m(a,d(b)) + \hbar\{a,b\}.
	\end{equation}
\end{proposition}

\begin{proof}
	By \cite[Proposition~7.2.1]{LodayVallette12}, the dg operad Koszul dual to the Batalin--Vilkovisky operad admits the following 	
	presentation
	\[
	\BV^! \cong \left(\mathcal{P}\left(s^{-1}\mathrm{S}\otimes_H E^*, s^{-2}\mathrm{S}\otimes_H(\q R)^{\perp}\right), (d_\varphi)^*\right).
	\]
	On the desuspension \(A\) of \(sA\), the linear duals \(\b^*\) and \(\m^*\) induce symmetric binary operations \(m\) and \(\{\, , \,\}\) of 
	respective degrees \(0\) and \(1\) and the linear dual \(\triangle^*\) induces a degree \(-2\) linear operator \(\tau\).
	The relations orthogonal to the quadratic analogues of the relations \(R\) given above asserts that \((A, m, \{\, , \,\})\) is a 
	Gerstenhaber algebra with which \(\tau\) commutes, that is a Gerstenhaber \(\mathbb{K}[\hbar]\)-algebra with \(|\hbar|=-2\).
	
	The induced differential \((d_\varphi)^*\) on the linear dual operad \(\left(\q\BV^{\ac}\right)^*\) is the unique derivation which sends 
	\(s^{-1}\b^*\) to \(-s^{-1}\triangle^* \circ s^{-1}\m^*\) and the other generators to \(0\). 
	This produces Relation~\eqref{eq:dKoszuldual} and the fact that the differential \(d\) commutes with the action of \(\hbar\) and with
	the commutative product. 
	The sign in Relation~\eqref{eq:dKoszuldual} comes from the fact that the differential on \(A\) is equal to the opposite of the 
	differential on \(sA\). 
\end{proof}

In short plain words, the desuspension of a \(\mathbb{K}\)-algebra over the operad \(\BV^!\) is a differential graded Gerstenhaber 
\(\mathbb{K}[\hbar]\)-algebra \((A, d, m, \{\, , \,\})\), with \(|\hbar|=-2\), except for the derivation relation with respect to the commutative product \(m\) which is replaced by
\[ [d, m]=\hbar \{\, , \,\}.\]

In the context of chiral algebras, Beilinson--Drinfeld introduced in \cite{BeilinsonDrinfeld04} an algebraic structure closely related to Koszul dual Batalin--Vilkovisky algebras. It admits at least two versions: one with a formal parameter \(\hbar\) in order to treat 
 perturbative Quantum Field Theories \cite{CG16, CG21} and one with a polynomial parameter \(\hbar\) enough to treat nonpeturbatively free theories \cite{GH18}. 

\begin{definition}\label{def:BDalg}
	A \emph{Beilinson--Drinfeld (BD) algebra} is a Gerstenhaber \(\mathbb{K}[\hbar]\)-algebra \((A, m ,\{\, , \,\} )\) on a chain 
	\(\mathbb{K}[\hbar]\)-complex  \((A,d)\), with \(|\hbar|=0\), such that the differential \(d\) is a derivation 
	with respect to \(\{\, , \,\}\) and such that 
	\begin{equation*}
	d(m(a,b)) = m(d(a),b) + (-1)^{|a|} m(a,d(b)) + \hbar\{a,b\}.
	\end{equation*}
	A \emph{complete BD-algebra} is defined similarly but over the ring of power series \(\KK[\![\hbar]\!]\). 
\end{definition}

To be self coherent, we used the homological degree convention in this definition, on the opposite to the abovementioned references where the Lie bracket has cohomological degree \(+1\). Let us denote by \(\BV_{-1}\) the operad encoding Batalin--Vilkovisky algebra with the operator \(\Delta\) and the Lie bracket placed in homological degree \(-1\). 

\begin{theorem}
	BD-algebras are suspensions of \(\BV_{-1}^!\)-algebras and complete BD-algebras are absolute \(\BV_{-1}^{\ac}\)-algebras. 
\end{theorem}

\begin{proof}
	The first point is a direct consequence of \cref{prop:DualBValg} with modified homological degrees accordingly. 
	For the notion of absolute algebras over a cooperad, we refer to \cite[Section~3]{RocaiLucio25}. The Koszul dual of the degree \(-1\) operator \(\Delta\) is a generating element of  degree \(0\) and arity \(1\) in the Koszul dual cooperad \(\BV_{-1}^{\ac}\). The arguments of Example~4.6 of \emph{loc. cit.} show that its action on an absolute \(\BV_{-1}^{\ac}\)-algebra is equivalent to an action of the ring of power series \(\KK[\![\hbar]\!]\). One concludes with the same arguments as in Section~4.2 of \emph{loc. cit.}.
\end{proof}

So BV-algebras are  Koszul dual to BD-algebras, up to a mild change of degree convention. 
Therefore the toolkit provided by the operadic calculus \cite[Chapter~11]{LodayVallette12} and \cite{RocaiLucio25} settles a bar-cobar adjunctions between categories of BV-(co)algebras and BD-(co)algebras, conilpotent or complete, which induce Quillen equivalences \cite{Vallette14}.
This gives a way to concretely relate these two types of algebraic structures and possibly explain duality phenomenon in Quantum Field Theories. 

\subsection{Koszul dual \(\cBV\)-algebras}
The translation the operadic statement of \cref{prop:cBV!} in terms of algebraic structure gives the following description for 
\(\cBV^!\)-algebras. 

\begin{proposition}\label{prop:DualcBValg}
	The desuspension of a differential graded \(\cBV^!\)-algebra structure on \(sA\) is 
		a Gerstenhaber \(\mathbb{K}[\hbar, \varepsilon]\)-algebra \((A, m ,\{\, , \,\} )\) on a chain 
	\(\mathbb{K}[\hbar]\)-complex  \((A,d)\), with \(|\hbar|=-2\) and \(|\varepsilon|=-1\), such that the differential \(d\) is a derivation 
	with respect to \(\{\, , \,\}\) and 
	\[d(\varepsilon a) + \varepsilon d(a)=-\hbar a.\]
	Moreover, it is equipped with a binary symmetric degree \(1\) nilpotent \(\mathbb{K}[\hbar, \varepsilon]\)-linear product \(\mu\)  satisfying
	\begin{itemize}
		\item[$\diamond$] the relation \(\varepsilon\mu(a,b)=0\), 
		
		\item[$\diamond$] the two products \(\mu\) and \(m\) commute, i.e. \(\mu \circ_i m = m \circ_i \mu\), for \(i=1,2\),
		and their composite is fully symmetric, i.e. \(\mu \circ_1 m = \left(\mu \circ_1 m\right)^{(23)}\),
		\item[$\diamond$] the product \(\mu\) and bracket \(\{\; ,  \, \}\) satisfy the Leibniz relation,
		\item[$\diamond$] the relation \(\varepsilon m(a,b)=\hbar\mu(a,b)\) holds,
		\item[$\diamond$] the derivative of the product \(\mu\) satisfies
		\begin{equation}\label{Eq:derivmu}
		d(\mu(a,b))=-\mu(d(a), b) - (-1)^{|a|}\mu(a, d(b)) + m(a,b) + \varepsilon\{a,b\} ~.
		\end{equation}
	\end{itemize}
\end{proposition}

Notice that the derivative of Relation~\eqref{Eq:derivmu} produces Relation~\eqref{eq:dKoszuldual}, which thus holds. 

\begin{proof}
	This is a direct corollary of  \cref{prop:cBV!}. The first three generators are the same ones as the Koszul dual operad \(\BV^!\). 
	The generator \(\square'=s^{-1}\square^*\) encodes a degree \(-1\) nilpotent operator \(\theta\)
	and the generator \(\c'=s^{-1}\c^*\) encodes a degree \(1\) symmetric nilpotent binary product \(n\). 
	The first homogenous quadratic relations of the operad \(\cBV^!\) coincide with the ones of the operad 
	\(\BV^!\) already described in \cref{prop:DualBValg} and the other ones produce the relations satisfied by \(\theta\) and \(\tau\). 
	The relation \([d, \theta] = \tau\) is a direct consequence of the definition of the internal differential \(d_1\), which implies
	\(d_1^*\left(\square'\right)=-\triangle'\) in \(\left(\q\cBV^{\ac}\right)^*\). 
	Similarly, the last relation comes from the fact that the differential \(d_\psi^*\) sends \(\c'\) to \(-\square' \circ \m'\) and that \(d_1^*\) sends
	\(\c'\) to \(-\b'\).
\end{proof}