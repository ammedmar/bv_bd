% !TEX root = ../ck2.tex

\subsection{Operads for strict algebras}\label{ss:strict_operads}

An operad \(\mathrm{O}\) is said to be \defn{presented} by data \((E, R)\), where \(E\) is an \(\Sy\)-module and \(R\) is an \(\Sy\)-submodule of the free operad \(\cT(E)\), if
\[
\mathrm{O} \cong \cP(E, R) \defeq \frac{\cT(E)}{(R)},
\]
with \((R)\) the operadic ideal generated by \(R\).

\subsubsection{The operad \(\Com\)}\label{def:com_operad}

We recall the standard quadratic presentation \((V,S)\) of the operad governing (dg) commutative algebras.
Here \defn{quadratic} means that \(V\) is an \(\Sy\)-module concentrated in homological degree \(0\) with trivial differential, and that \(S\) is an \(\Sy\)-submodule of \(\cT(V)^{(2)}\), the part of the free operad \(\cT(V)\) spanned by all compositions involving exactly two generators.
As we review in \cref{ss:koszul_duals}, quadratic presentations are the classical setting where Koszul duality can be applied.

\medskip\noindent The operad \defn{\(\Com\)} is presented by the following quadratic data \((V, S)\), where the symmetric action on the generator is trivial:

\medskip\noindent \(\diamond\) \(V \defeq \KK\set{\m}\) where \(\m \defeq \M\) with degree \(\bars{\m}=0\),

\medskip\noindent \(\diamond\) \(S\) is the \(\Sy\)-module spanned by the relator
\vspace*{-7pt}
\[
\textsc{associativity: } \LL{1}{2}{3} - \RR{1}{2}{3},
\]
that is
\[
\Com \defeq \frac{\mathcal{T}(\m)}{\big(\text{associativity}\big)}.
\]

\subsubsection{The operad \(\BV\)}\label{def:OperadBV}

A straightforward definition of the operad \defn{\(\BV\)} governing (dg) \(\BV\)-algebras is
\[
\BV \defeq \frac{\Com \vee \mathcal{T}(\triangle)}{\big(\triangle \circ \triangle,\ \triangle \ \text{second-order}\big)}
\]
or, explicitly,
\[
\BV \cong  \frac{\mathcal{T}(\m, \triangle)}{\big(\text{associativity}, \triangle \circ \triangle,\ \triangle \ \text{second-order}\big)},
\]
where the generators are in degree \(0\) and the differential is trivial.

\medskip Unfortunately, the second-order relation involves the composition of \textit{three} generators, making the associated presentation unfit fit for Koszul duality techniques.
To find a better suited presentation, recall that in a  \(\BV\)-algebra \((A, d, \m, \triangle)\) the failure of \(\triangle\) to be a derivation of the product defines a shifted Lie bracket \(\b \defeq [\triangle, \m]\) for which the Leibniz relation with respect to \(\m\) is equivalent to the second-order relation for \(\triangle\).
This leads to the following alternative presentation, which is \defn{inhomogeneous quadratic}: here the generating \(\Sy\)-module has trivial differential and the relators lie in \(\cT(E)^{(1)} \oplus \cT(E)^{(2)}\), that is, a mixture of linear and quadratic terms in the free operad \(\cT(E)\).

\begin{lemma}\label{lemma:presentationBV}
	The operad \(\BV\) is presented by \((E, R)\), where the symmetric action on all generators is trivial:

	\medskip\noindent \(\diamond\) \(E \defeq \KK\set{\m} \oplus \KK\set{\b} \oplus \KK\set{\triangle}\) where
		\[
		\m \defeq \M,\quad \b \defeq \B,\quad \triangle \defeq \D~,
		\]
		with \(|\m| = 0\) and \(|\triangle| = |\b| = 1\).

	\medskip\noindent \(\diamond\) \(R\) is spanned, as an \(\Sy\)-module, by the following 6 relations:

	\noindent
	\begin{tabular}{ll}
		\textnormal{\textsc{associativity:}} \(\LL{1}{2}{3} - \RR{1}{2}{3}\) &
		\textnormal{\textsc{square-zero:\quad}} \(\DD\) \\
		\textnormal{\textsc{bracket:}} \(\BB{1}{2} - \hspace*{-5pt} \DM{1}{2} \hspace*{-5pt} + \MDL{1}{2} + \MDR{1}{2}\) &
		\textnormal{\textsc{leibniz:}} \(\LBM{1}{2}{3}- \hspace*{-5pt} \LMB{1}{3}{2}-\RMB{1}{2}{3}\)\\
		\textnormal{\textsc{jacobi:}} \(\BBB{1}{2}{3}+ \hspace*{-5pt} \BBB{2}{3}{1}+ \hspace*{-5pt} \BBB{3}{1}{2}\) &
		\textnormal{\textsc{derivation:}} \(\DB{1}{2} - \BDL{1}{2} - \BDR{1}{2}\)~.
	\end{tabular}
\end{lemma}

\begin{proof*}
	Leibniz relation for \(\b\) is equivalent to the second-order relation for \(\triangle\), while the Jacobi relation and the derivation relation are direct consequences of the other relations.
\end{proof*}

\begin{remark}
	As noted above, the Jacobi and derivation relations follow from the other relations, but we include them explicitly since a presentation of this form is required for applying the theory of Koszul duality to the operad \(\BV\) (see \cref{ex:qlCondition}).
\end{remark}

\subsubsection{The operad \(\cBV\)}

The operad \defn{\(\cBV\)} governing \(\cBV\)-algebras is
\[
\BV \defeq \dfrac{\BV \vee \mathcal{T}(\square)}
{\big([\triangle, \square], \ \square \ \text{second-order}\big)}
\]
or, explicitly,
\[
\cBV \defeq \frac{\mathcal{T}(\m, \triangle, \square)}
{\big(\text{associativity}, \triangle \circ \triangle,\ [\triangle, \square],\ \triangle\text{ second-order},\ \square\text{ second-order}\big)},
\]
where the differential is the unique derivation extending
\[
\dif\, \triangle \defeq \square,
\qquad
\dif\, \m \defeq 0.
\]

%\begin{remark}
%	A generator for \(\square\) is not required in the definition of a \(\cBV\)-algebra (\cref{def:cbv_algebra}), since it is automatically produced from the generator for \(\triangle\); explicitly, it arises as the obstruction \(n\).
%	However, it must appear as a generator of the operad \(\cBV\): without it, the image of \(\triangle\) under the operadic differential \(\dif\) would necessarily vanish, forcing \(\triangle\) to commute with the underlying differential of every \(\cBV\)-algebra.
%	Similarly, the relations expressing that \(\square\) commutes with \(\triangle\) and is a second–order operator hold automatically in any \(\cBV\)-algebra, but they must be imposed in the defining ideal of the operad so that the differential \(\dif\) is well defined.
%\end{remark}

As for the first presentation given for the operad \(\BV\), this presentation of \(\cBV\) is not well suited to our purposes.
We introduce new generators corresponding to \([\triangle,\m]\) and \([\square,\m]\) and describe the operad \(\cBV\) via a \defn{differential graded inhomogeneous quadratic} presentation, i.e., one in which the generating \(\Sy\)-module \(E\) may carry a non-trivial differential and, as before, the relators lie in \(\cT(E)^{(1)} \oplus \cT(E)^{(2)}\).

\begin{lemma}\label{lemma:presentationcBV}
	The operad \(\cBV\) is presented by \((E_\bullet, R_\bullet)\), where the symmetric action on all generators is trivial:

	\medskip\noindent \(\diamond\) \(E_\bullet \defeq \big(\KK\set{\m} \oplus \KK\set{\b} \oplus \KK\set{\triangle} \oplus \KK\set{\c} \oplus \KK\set{\square},\ \dif\big)\) where
	\[
	\m \defeq \M,\quad
	\b \defeq \B,\quad
	\triangle \defeq \D,\quad
	\c \defeq \C,\quad
	\square \defeq \BOX,
	\]
	with \(|m| = |c| = |\square| = 0\) and \(|b| = |\triangle| = 1\), and differential
	\[
	\dif \colon \M \mapsto 0,\quad
	\B \mapsto \C \mapsto 0,\quad
	\D \mapsto \square \mapsto 0.
	\]

	\medskip\noindent \(\diamond\) \(R_\bullet\) is spanned as an \(\Sy\)-module by the following 11 relators.\anibal{I count 11?}

	\noindent
	\begin{tabular}{ll}
		\textnormal{\textsc{associativity:}} \(\LL{1}{2}{3} - \RR{1}{2}{3}\) &
		\textnormal{\textsc{square-zero:\quad}} \(\DD\) \\[4pt]

		\textnormal{\textsc{bracket \(\triangle\):}} \(\BB{1}{2} - \hspace*{-5pt}\DM{1}{2}\hspace*{-5pt} + \MDL{1}{2} + \MDR{1}{2}\) &
		\textnormal{\textsc{leibniz \(\triangle\):}} \(\hspace*{-5pt}\LBM{1}{2}{3}-\hspace*{-5pt}\LMB{1}{3}{2}-\RMB{1}{2}{3}\) \\[4pt]

		\textnormal{\textsc{bracket \(\square\):}} \(\CC{1}{2} - \hspace*{-5pt}\BoxM{1}{2}\hspace*{-5pt} + \MBoxL{1}{2} + \MBoxR{1}{2}\) &
		\textnormal{\textsc{leibniz \(\square\):}} \(\hspace*{-5pt}\LBoxM{1}{2}{3}-\hspace*{-5pt}\LMBox{1}{3}{2}-\RMBox{1}{2}{3}\) \\[4pt]

		\textnormal{\textsc{jacobi:}} \(\hspace*{-5pt}\BBB{1}{2}{3}+\hspace*{-5pt}\BBB{2}{3}{1}+\hspace*{-5pt}\BBB{3}{1}{2}\) &
		\textnormal{\textsc{derivation:}} \(\hspace*{-5pt}\DB{1}{2} - \BDL{1}{2} - \BDR{1}{2}\)
	\end{tabular}

	\noindent
	\begin{tabular}{l}
		\textnormal{\textsc{compatibility 1:}} \hspace{6pt}\(\DBox - \BoxD\) \\[4pt]
		\textnormal{\textsc{compatibility 2:}} \(\CB{1}{2}{3}+ \hspace*{-5pt}\CB{2}{3}{1}+ \hspace*{-5pt}\CB{3}{1}{2}- \hspace*{-5pt}\BC{1}{2}{3}- \hspace*{-5pt}\BC{2}{3}{1}- \hspace*{-5pt}\BC{3}{1}{2}\) \\[4pt]
		\textnormal{\textsc{compatibility 3:}} \(\BoxB{1}{2} \hspace*{-5pt}-\hspace*{-5pt} \DC{1}{2} \hspace*{-5pt}- \CDL{1}{2} +\BBoxL{1}{2}-\CDR{1}{2}+\BBoxR{1}{2}\)~.
	\end{tabular}
\end{lemma}

\begin{proof*}
	First one can see that both \(E_\bullet\) and \(R_\bullet\) are stable under the differential \(\dif\), so this differential graded quadratic-linear data is well defined.
	The rest of the proof is similar to the one for the operad \(\BV\):
	the Leibniz relation for \(\c\) is equivalent to the second-order relation for \(\square\) and the compatibility relations 2 and 3 are direct consequences of the other relations.
\end{proof*}

\begin{remark}
	Like for the operad \(\BV\), the Jacobi relation, the derivation relation, and the compatibility relations 2 and 3 are not mandatory to obtain an equivalent presentation for the operad \(\cBV\), but we need to consider all of them in order to get a presentation suitable for the theory of Koszul duality developed in \cref{subsubsec:QLcBV}.
\end{remark}

\begin{remark}
	Notice that the “bracket’’ \(\c\) induced by the second–order operator \(\square\) need not satisfy the Jacobi relation, since \(\square\) does not square to zero.
	For the same reason, \(\square\) is not, in general, a derivation of the bracket \(\c\).
\end{remark}