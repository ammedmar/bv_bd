% !TEX root = ../bv_bd.tex

\section{Preliminaries}

\subsection{Conventions}

We work over a field \(\KK\) of characteristic \(0\) and over the underlying category of differential graded modules.
We use the homological degree convention, for which differentials have degree \(-1\) (so cohomology lies in degree \(-n\)).
We denote the symmetric groups by \(\Sy_n\).
We use the operadic conventions introduced in \cite{LodayVallette12}.

\subsection{Degree involution}

Let \(\rT\colon \gVec \to \gVec\) be the \defn{degree involution} given on objects by \((\rT V)_n \coloneqq V_{-n}\).
This functor is a symmetric monoidal autoequivalence, and therefore it acts on operads over \(\gVec\).
For any operad \(\rO\) in ??, the degree involution functor induces a equivalence of categories
\[
\begin{tikzcd}[column sep=small, row sep=-3pt]
	\rO\text{-}\mathsf{Alg}
	\arrow[r] & \rT(\rO)\text{-}\mathsf{Alg} \\
	A \arrow[r, maps to] & \rT(A).
\end{tikzcd}
\]

The degree involution functor is also an equivalence of categories between chain complexes and cochain complexes.


\subsection{\(n\)-Poisson algebras}

The homology of the little \(n\)-disk operad for \(n \geq 2\) is the operad \(\Pois_n\), which can be presented by \((E, R)\), where the symmetric action on all generators is trivial:\anibal{Is this action corect?}

\medskip\noindent \(\diamond\) \(E \defeq \KK\set{\m} \oplus \KK\set{\b}\) where
\[
\m \defeq \M,\quad \b \defeq \B~,
\]
with \(|\m| = 0\) and \(|\b| = 1 - n\).

\medskip\noindent \(\diamond\) \(R\) is spanned, as an \(\Sy\)-module, by the following 3 relations:

%	\noindent
\begin{tabular}{ll}
	\textnormal{\textsc{associativity:}} & \(\LL{1}{2}{3} - \RR{1}{2}{3}\) \\
	\textnormal{\textsc{jacobi:}} & \(\BBB{1}{2}{3}+ \hspace*{-5pt} \BBB{2}{3}{1}+ \hspace*{-5pt} \BBB{3}{1}{2}\) \\
	\textnormal{\textsc{leibniz:}} & \(\LBM{1}{2}{3}- \hspace*{-5pt} \LMB{1}{3}{2}-\RMB{1}{2}{3}\)~.
\end{tabular}

\medskip Explicitly, an \defn{\(n\)-Poisson algebra} in \(\gVec\), i.e., an operad map \(\Pois_n \to \End_A\), is a triple \((A, m, b)\) where \(m \colon A \otimes A \to A\) is a graded commutative and associative product and \(b \colon A \otimes A \to A\) is a Lie bracket of degree \(1-n\) satisfying:
\[
b(a, m(x,y)) = m(b(a,x), y) + (-1)^{(|a| + 1 - n)|x|} m(x, b(a,y))
\]
for all homogeneous \(a, x, y \in A\).

An \(n\)-Poisson algebra on chain complexes is the same data with the additional constraint that \(m\) and \(b\) are closed in \(\Hom_A\).

\medskip\noindent We will be mostly interested in the operads \(\Pois_0\) and \(\Pois_2 \cong \rT\Pois_0\), controlling so-called \defn{Gerstenhaber algebras}.

\begin{example}
	Consider a finite-dimensional smooth manifold \(M\).
	The graded vector space \(\Sym^\bullet \rT M\) of \defn{polyvector fields} carries a \(\Pois_0\)-algebra structure when graded non-positively.
	The bracket \(\{-,-\}\) is given by the extension of the Lie bracket of vector fields.

	We can think of this as the ring of functions on the shifted cotangent bundle \(\rT^*[-1] M\), in the supergeometric sense.
	The \(\Pois_0\) structure on \(\rT^*[-1] M\) is a supersymetric generalization of the Poisson structure on the functions on the cotangent bundle of \(M\)

	Choosing a function \(S \colon M \to \R\) satisfying the \textit{classical master equation} \(\{S,S\} = 0\) we can promote this \(\Pois_0\)-algebra to \(\dgVec\) with differential \(d = \{S, -\}\).
	Its underlying cochain complex is a model for the derived critical locus of \(S\) \cite[\S?]{CostelloGwilliam2021FactorizationAlgebrasQFTv2}.
\end{example}

\begin{example}
	For a bounded cochain complex \((V, d_V)\) consider the \textit{functions on its shifted cotangent bundle}
	\[
	\cO(\rT^*[-1] V) \defeq \big(\Sym(V^\vee \ot V[1]), d\big).
	\]
	We can define a \(\Pois_0\) structure on this complex by extending the evaluation map using the Leibniz relation.
\end{example}

\subsection{Quantisation of \(\Pois_0\)-algebras}

A \defn{Beilinson--Drinfeld algebra} or \defn{\(\BD\)-algebra} \((A^q, d,\{-,-\})\) is a graded commutative algebra \(A^q\), flat as a module over \(\KK[[\hbar]]\), equipped with a degree \(1\) Poisson bracket \(\{-,-\}\) and an \(\KK[[\hbar]]\)-linear differential \(d\), such that for all homogeneous \(a,b\in A\) one has
\[
d(ab)=(da)b+(-1)^{\lvert a\rvert}a(db)+\hbar\{a,b\}.
\]

\medskip\noindent Given a BD-algebra \(A^q\), we can restrict to \(\hbar=0\) by setting
\[
A^q_{\hbar=0}\coloneqq A\otimes_{\KK[[\hbar]]}\KK[[\hbar]]/(\hbar).
\]
The induced differential on \(A_{\hbar=0}\) is a derivation, hence \(A^q_{\hbar=0}\) is a \(P_0\)-algebra.

\medskip\noindent A \defn{\(\BV\)-quantisation} of a \(P_0\)-algebra \(A\) is a BD algebra \(A^q\) such that \(A^q_{\hbar=0} = A\).

\begin{example}
	\(\R^n\) ...
\end{example}

%Likewise, we can restrict to \(\hbar\neq 0\) by setting
%\[
%A_{\hbar\neq 0}\coloneqq A^q\otimes_{\KK[[\hbar]]}\KK((\hbar)).
%\]
%In this case we obtain only a cochain complex, and its cohomology need not inherit an algebra structure, unlike \(H^*(A_{\hbar=0})\).

\subsection{Koszul duality}

\begin{example}
	As explained in \cite[\S13.3.16]{LodayVallette12}, we have
	\[
	\Pois_n^! \cong \mathrm s^n\Pois_n.
	\]
\end{example}

\subsection{Algebras over cooperads}

Review of \cite{rocailucio2025absolute}.

\subsection{Extending scalars}

We will be interested in algebras over the category of graded \(\KK[[h\hbar]]\)-modules.
Given an operad \(\rO\) over \(\gVec\), an \(\rO{[\hbar]}\)-algebra is ...



