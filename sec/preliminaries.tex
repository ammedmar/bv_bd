% !TEX root = ../bv_bd.tex

\section{Preliminaries}

\subsection{Conventions}

We work over the symmetric monoidal category of cochain complexes of \(\KK\)-vector spaces with \(\KK\) a field of characteristic \(0\).
We use the operadic conventions of \cite{LodayVallette12} with the understanding that those are written for chain complexes.

\subsection{Degree involution}

Let \(\rT\colon \gVec \to \gVec\) be the \defn{degree involution} given on objects by \((\rT V)_n \defeq V_{-n}\).
This functor is a symmetric monoidal autoequivalence, and therefore it acts on operads over \(\gVec\).
For any operad \(\rO\) in ??, the degree involution functor induces a equivalence of categories
\[
\begin{tikzcd}[column sep=small, row sep=-3pt]
	\rO\text{-}\mathsf{Alg}
	\arrow[r] & \rT(\rO)\text{-}\mathsf{Alg} \\
	A \arrow[r, maps to] & \rT(A).
\end{tikzcd}
\]

The degree involution functor is also an equivalence of categories between chain complexes and cochain complexes.