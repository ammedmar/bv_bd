% !TEX root = ../ck2.tex

\subsection{Application to \(\cBV\)}\label{ss:cbv}

\subsubsection{Quadratic-linear conditions for \(\cBV\)}\label{subsubsec:QLcBV}

\begin{lemma}\label{lemme:qlsforcBV}
	The inhomogeneous quadratic presentation \((E_\bullet, R_\bullet)\) of the operad \(\cBV\) satisfies the quadratic-linear conditions.
\end{lemma}

\begin{proof*}
	Condition \((ql_1)\) is trivially satisfied.
	In order to establish Condition \((ql_2)\), we have to compute
	\[
	\{R_\bullet \circ_{(1)} E_\bullet + E_\bullet \circ_{(1)} R_\bullet\} \cap \cT(E_\bullet)^{(2)};
	\]
	the only way to possibly get non-trivial elements is to consider either the inhomogeneous quadratic relations \textsc{bracket} \(\triangle\) or the \textsc{bracket} \(\square\) in \(R_\bullet\) composed above or below with the generators \(\m\), \(\b\), \(\triangle\), \(\c\), or \(\square\) in \(E_\bullet\).
	This gives the following 8 possibilities that are all straightforward to check: in the table below the \emph{inhomogeneous relation} is composed with a \emph{generator} to produce a \emph{combination} which either produces \emph{no relation}, is equal to a \emph{homogeneous relation}, or is a \emph{direct consequence} of a homogeneous relation.
	\begin{center}
		\renewcommand{\arraystretch}{1.15}
		\begin{tabular}{c l l l l }
			\hline
			& Inhomogeneous Relation & Generator & Combination & Homogeneous Relation \\
			\hline
			1 & \textsc{bracket \(\triangle\)} & \(\m\) & \begin{minipage}[t]{3.1cm} \(\b \circ_1 \m + \m \circ_1 \b=\) \\
				\(\big(\b \circ_1 \m + \m \circ_1 \b\big)^{(123)}\) \end{minipage} & Consequence of \textsc{leibniz \(\triangle\)} \\
			2 & \textsc{bracket \(\triangle\)} & \(\b\) & \textsc{jacobi \(\triangle\)} & \textsc{jacobi \(\triangle\)} \\
			3 & \textsc{bracket \(\triangle\)} & \(\triangle\) & \textsc{derivation} & \textsc{derivation} \\
			4 & \textsc{bracket \(\square\)} & \(\m\) & \begin{minipage}[t]{3.1cm} \(\c \circ_1 \m + \m \circ_1 \c=\) \\
				\(\big(\c \circ_1 \m + \m \circ_1 \c\big)^{(123)}\) \end{minipage} & Consequence of \textsc{leibniz \(\square\)} \\
			5 & \textsc{bracket \(\triangle\)} and \textsc{bracket \(\square\)} & \(\c\) and \(\b\) & \textsc{compatibility 2} & \textsc{compatibility 2} \\
			6 & \textsc{bracket \(\triangle\)} and \textsc{bracket \(\square\)} & \(\square\) and \(\triangle\) & \textsc{compatibility 3} & \textsc{compatibility 3} \\
			7 & \textsc{bracket \(\square\)} & \(\c\) & No relation & No relation \\
			8 & \textsc{bracket \(\square\)} & \(\square\) & No relation & No relation \\
			\hline
		\end{tabular}
	\end{center}
	Let us make explicit two types of computations to show how it works.
	In the first case of the above list, one comes up with the following six elements of \(\cT(E_\bullet)\):
	\begin{align*}
		\Theta(a,b,c) &\defeq \LBM{a}{b}{c}-\DLMM{a}{b}{c}+\LMDM{a}{b}{c}+\LMMBBB{a}{b}{c}~,\\
		\Xi(a,b,c) &\defeq \LMB{a}{b}{c}-\LMDM{a}{b}{c}+\LMMB{a}{b}{c}+\LMMBB{a}{b}{c}~,
	\end{align*}
	where \((a,b,c)\) is either equal to \((1,2,3)\), \((2,3,1)\), or \((3,1,2)\).
	The only cubical trees (i.e.\ with three vertices) on the right-hand sides which appear in a composite of a homogeneous quadratic relation (associativity here) in \(R_\bullet\) with a generator (\(\triangle\) here) in \(E_\bullet\) are the ones where the operator \(\triangle\) sits at the top or at the bottom of the composite of two products \(\m\).
	So, up to the associativity relation composed with one \(\triangle\), the only way to get ony quadratic terms from a linear combination of the aforementioned six terms is
	\begin{align*}
		\Theta(1,2,3)+\Xi(1,2,3)-\Theta(2,3,1)+\Xi(2,3,1)
		&=
		\LBM{1}{2}{3}+\LMB{1}{2}{3}-\LBM{2}{3}{1}-\LMB{2}{3}{1},
	\end{align*}
	and the similar elements obtained by cyclic permutations.
	This homogeneous quadratic relation is equal to the difference of the Leibniz \(\triangle\) relation with inputs \((1,2,3)\) and the Leibniz \(\triangle\) relation with inputs \((2,3,1)\), so it is not new.

	In the last case of the above list, one comes up with the following three elements of \(\cT(E_\bullet)\):
	\begin{align*}
		&\CBoxL{1}{2}-\BoxMBoxL{1}{2}+\MBoxBoxL{1}{2}+\MBoxBox{1}{2}~,\\
		&\CBoxR{1}{2}-\BoxMBoxR{1}{2}+\MBoxBox{1}{2}+\MBoxBoxR{1}{2}~,\\
		&\BoxC{1}{2}-\BoxBoxM{1}{2}+\BoxMBoxL{1}{2}+\BoxMBoxR{1}{2}~.
	\end{align*}
	None of the cubical trees on the right-hand sides appear in any composite of a homogeneous quadratic relation in \(R_\bullet\) with a generator in \(E_\bullet\).
	Then, the presence of the three different cubical trees with two adjacent generators \(\square\) prevents us from producing a non-trivial homogeneous quadratic relation from a linear combination of these above three terms.
\end{proof*}

\subsubsection{The cooperad \(\cBV^{\ac}\)}

In order to develop an effective theory of \(\cBV_\infty\)-algebras, we will make explicit the cooperad
\[
\cBV^{\ac} \defeq \big(\q\cBV^{\ac}, d_1 + d_{\psi}\big)
= \big(\cC(sE_\bullet, s^2 \q R_\bullet), d_1 + d_{\psi}\big).
\]
To do so we start with an explicit description of the operad \(\cBV^! \).
%\[
%\cBV^! \defeq \mathrm{S} \otimes_H \big(\cBV^{\ac}\big)^*.
%\]

\begin{lemma}\label{prop:cBV!}
	The operad \(\cBV^!\) admits the following presentation, where the symmetric action on all generators is trivial:

	\medskip\noindent \(\diamond\) The generating \(\Sy\)-module is given by
	\[
	\Big(\KK\set{\m'} \oplus \KK\set{\c'} \oplus \KK\set{\triangle'} \oplus \KK\set{\b'} \oplus \KK\set{\square'}\ ,\ (d_1)^* + (d_\psi)^*\Big),
	\]
	where
	\[
	\m' \defeq \Mprime,\quad
	\b' \defeq \Bprime,\quad
	\triangle' \defeq \Dprime,\quad
	\c' \defeq \Cprime,\quad
	\square' \defeq \BOXprime,
	\]
	with
	\[
	|\m'| = |\c'| = 0, \quad
	|\b'| = |\square'| = -1, \quad
	|\triangle'| = -2.
	\]
	The differential \((d_1)^*\) is the unique derivation extending the assignment
	\[
	\Mprime \mapsto 0,\quad
	\Cprime \mapsto -\Bprime \mapsto 0,\quad
	\BOXprime \mapsto -\Dprime \mapsto 0,
	\]
	and the differential \((d_\psi)^*\) is the unique derivation extending the assignment
	\[
	\Bprime \mapsto -\DMprime{1}{2},\quad
	\Cprime \mapsto -\BoxMprime{1}{2}.
	\]

	\medskip\noindent \(\diamond\) The \(\Sy\)-module of relations is spanned by the following 11 relators:

	\begin{tabular}{ll}
		\textnormal{\textsc{jacobi:}}
		\(\LLprime{1}{2}{3}+\LLprime{2}{3}{1}+\LLprime{3}{1}{2}\) &
		\textnormal{\textsc{associativity:}}
		\(\BBBprime{1}{2}{3}-\BBRprime{1}{2}{3}\) \\[4pt]

		\textnormal{\textsc{leibniz \(\m'\text{--}\b'\):}}
		\(\LMBprime{1}{2}{3}-\LBMprime{1}{3}{2}-\LBMprime{2}{3}{1}\) & \\[4pt]

		\textnormal{\textsc{commutativity \(\m'\text{--}\triangle'\):}}
		\(\DMprime{1}{2}-\MDLprime{1}{2}\) &
		\textnormal{\textsc{commutativity \(\b'\text{--}\triangle'\):}}
		\(\DBprime{1}{2}-\BDLprime{1}{2}\) \\[4pt]

		\textnormal{\textsc{nilpotency:}}
		\(\BoxBoxprime,\ \BoxCprime{1}{2},\ \CBoxLprime{1}{2},\ \LCCprime{1}{2}{3}\) &
		\textnormal{\textsc{commutativity \(\triangle'\text{--}\square'\):}}
		\(\DBoxprime - \BoxDprime\) \\[4pt]

		\textnormal{\textsc{leibniz \(\m'\text{--}\square'\):}}
		\(\LMBoxprime{1}{2}{3}+\LBoxMprime{1}{3}{2}+\LBoxMprime{2}{3}{1}\) &
		\textnormal{\textsc{commutativity \(\m'\text{--}\square'\):}}
		\(\BoxMprime{1}{2}+\MBoxLprime{1}{2}\)
	\end{tabular}

	\noindent
	\begin{tabular}{l}
		\textnormal{\textsc{commutativity \(\b'\text{--}\c'\):}}
		\(\CBprime{1}{2}{3}-\CBprime{2}{3}{1},\
		\BCprime{1}{2}{3}-\BCprime{2}{3}{1},\
		\CBprime{1}{2}{3}-\BCprime{1}{2}{3}\) \\[4pt]

		\textnormal{\textsc{compatibility \(\b'\square' \text{--} \c'\triangle'\):}}
		\(\BoxBprime{1}{2}-\DCprime{1}{2},\
		\DCprime{1}{2}+\CDLprime{1}{2},\
		\BoxBprime{1}{2}+\BBoxLprime{1}{2}\)~.
	\end{tabular}
\end{lemma}

\begin{proof*}
	This proof is similar and extends the one of the \(\BV\) case given in \cref{prop:DualBValg}: we apply \cite[Proposition~7.2.1]{LodayVallette12} to get the following presentation
	\[
	\big(\q\cBV^{\ac}\big)^* \cong \mathcal{P}\big(s^{-1}(E_\bullet)^*, s^{-2}(\q R_\bullet)^{\perp}\big).
	\]
	Here the notation with a prime stands for the desuspension of the linear dual: \(\m' \defeq s^{-1}\m^*\), for instance.
	The rest is a straightforward computation of the orthogonal space of quadratic relations.
	The dual differentials are direct consequences of the definitions of the coderivations on the Koszul dual cooperad; the minus sign comes from the Koszul convention \(\big(s^{\otimes 2}\big)^* = -s^* \otimes s^*\).
\end{proof*}

\begin{lemma}
	MISSING LEMMA\anibal{missing lemma about the decomposition of BV as an S-module (I think)}
\end{lemma}

\begin{theorem}\label{thm:FormcBVac}
	The cooperads \(\BV^\ac\) and \(\cBV^\ac\) admit the following decompositions as cooperads:
	\[
	\begin{tikzcd}[column sep=-2.5pt]
		\BV^\ac & \cong & \Com^\ac & \oplus &
		\rM^*
		\arrow[loop below, in=240, out=300, distance=2.3em, "d_\varphi"]
	\end{tikzcd}
	\quad\text{and}\quad
	\begin{tikzcd}[column sep=-2.5pt]
		\cBV^\ac & \cong & \Com^\ac & \oplus &
		\rM^*
		\arrow[rr, bend left=45, "\quad d_1 = s^{-1}", out=80, in=100, distance=1.1em]
		\arrow[loop below, in=240, out=300, distance=2.3em, "d_\varphi"]  &
		\oplus & s^{-1}\rM^* \arrow[loop below, in=245, out=295, distance=2em, "-d_\varphi"]
	\end{tikzcd}
	\]
	where
	\begin{itemize}[label=\(\diamond\),  leftmargin=*]
		\item \(\Com^\ac\) is the sub-cooperad spanned by the cogenerator \(s\m\).
		\item \(\rM^*\) is the coideal of \(\BV^\ac\) spanned by the cogenerators \(s\b\) and \(s\triangle\).
		\item \(s^{-1}\rM^*\) is the coideal of \(\cBV^\ac\) spanned by the cogenerators \(s\c\) and \(s\square\).
	\end{itemize}
\end{theorem}

\begin{proof*}
	Recall that the quadratic analogue \(\q\BV\) of the operad \(\BV\) is given by a distributive law, see \cite[Section~1.2]{GCTV12}.
	This implies that the underlying graded \(\Sy\)-modules of the Koszul dual cooperad \(\q\BV^{\ac}\) is given by
	\[
	\q\BV^{\ac} \cong \mathrm{D}^{\ac} \circ \Lie_1^{\ac} \circ \Com^{\ac},
	\]
	where \(\mathrm{D} \defeq T(\triangle)/\big(\triangle^2\big)\) is the algebra of dual numbers and \(\Lie_1\) is the operad of shifted Lie algebras with bracket of degree \(1\).
	So the linear dual operad \(\big(\q\BV^{\ac}\big)^*\) splits as
	\[
	\big(\q\BV^{\ac}\big)^* \cong \mathrm{S}\Lie \oplus \rM,
	\]
	where \(\mathrm{S}\Lie\) is the sub-operad spanned by the generator \(\m'\) and where \(\rM\) is the \(\mathrm{S}\Lie\)-bimodule given by the ideal of \(\big(\q\BV^{\ac}\big)^*\) spanned by the two generators \(\b'\) and \(\triangle'\).

	\medskip

	The presentation of the operad \(\big(\q\cBV^{\ac}\big)^*\) given in the above \cref{prop:cBV!} shows that the pair of nilpotent generators \(\c'\) and \(\square'\) behave in the same way as the pair of generators \(\b'\) and \(\triangle'\) up to suspension.
	This implies the following isomorphism of operads
	\[
	\big(\q\cBV^{\ac}\big)^* \cong \mathrm{S}\Lie \oplus \rM \oplus s\rM,
	\]
	where \(s\rM\) is isomorphic to the ideal of \(\big(\q\cBV^{\ac}\big)^*\) spanned by
	the two generators \(\c'\) and \(\square'\).
	Dually, we get the isomorphism of cooperads
	\[
	\q\cBV^{\ac} \cong \Com^\ac \oplus \rM^* \oplus s^{-1}\rM^*.
	\]
	Under this isomorphism, the codifferential \(d_1\) is equal to the desuspension isomorphism \(s^{-1} \colon \rM^* \to s^{-1}\rM^*\).
	Since the total number of cogenerators \(s\c\) and \(s\square\) is preserved under the codifferential \(d_\psi\), both coideals \(\rM^*\) cogenerated by \(s\b\) and \(s\triangle\) and \(s^{-1}\rM^*\) cogenerated by \(s\c\) and \(s\square\) are stable under \(d_\psi\), where they are respectively equal to \(d_\varphi\) and \(-d_\varphi\).
\end{proof*}

\subsubsection{Koszul property}

\begin{theorem}\label{prop:cBVKoszul}
	The operad \(\cBV\) is Koszul.
\end{theorem}

\begin{proof*}
	We have already seen in \cref{lemme:qlsforcBV} that the dg inhomogeneous quadratic data \(\big(E_\bullet, R_\bullet\big)\)
	introduced in \cref{lemma:presentationcBV} for the operad \(\cBV\) satisfies the quadratic conditions.
	It remains to study its homology twisting morphism \(\bar{\kappa}\).
	By \cref{thm:FormcBVac}, the homology of the Koszul dual cooperad \(\cBV^\ac\) with respect to the internal codifferential
	\(d_1\) is isomorphic to the Koszul dual cooperad \(\Com^\ac\):
	\[
	H_\bullet\big(\cBV^{\ac}, d_1\big) \cong \Com^\ac.
	\]
	\cref{thm:FormcBVac} also shows that the induced codifferential \(\bar{d}_\psi = 0\) is trivial in the case of the operad \(\cBV\).
	\cref{thm:Homology} computes the homology operad of \(\cBV\):
	\[
	H_\bullet\big(\cBV, \partial\big) \cong \Com.
	\]
	In the end, the homology twisting morphism \(\bar{\kappa}\) is equal to the canonical twisting morphism
	\(\Com^\ac \to \Com\) of the operad \(\Com\), which is known to be Koszul, see \cite[Proposition~13.1.2]{LodayVallette12}.
\end{proof*}

\begin{remark}
	In the present case of the operad \(\cBV\), an even stronger phenomenon appears, which does not hold in full generality: the homology of the operad \(\cBV\) and the homology of the Koszul dual cooperad \(\cBV^\ac\) are given by the homology of their presentations
	\begin{align*}
		&H_\bullet\big(\cBV, \partial\big) \cong
		\mathcal{P}\big(H_\bullet(E_\bullet, \partial), H_\bullet(R_\bullet, \partial)\big) \cong \Com
		\quad \text{and} \quad \\
		&H_\bullet\big(\cBV^\ac, d_1\big) \cong
		\mathcal{C}\big(sH_\bullet(E_\bullet, \partial), s^2 H_\bullet(qR_\bullet, \partial)\big) \cong \Com^\ac~.
	\end{align*}
\end{remark}