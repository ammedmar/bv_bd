% !TEX root = ../ck2.tex

\subsubsection{Koszul resolution}\label{ss:cofibrant_operads}

Homotopy \(\rO\)-algebras are defined as algebras over a \defn{cofibrant replacement} of the operad \(\rO\), that is, a cofibrant operad \(\rQ\rO\) with an acyclic fibration
\(
\begin{tikzcd}[column sep=15pt]
	\rQ\rO \arrow[r, ->>, "\sim\ "] & \rO.
\end{tikzcd}
\)
A \defn{Koszul resolution} is a cofibrant replacement of the specific form
\begin{equation}\label{eq:Omega Oac->O}
	\begin{tikzcd}[column sep=22pt]
		\Omega\rO^{\ac} \arrow[r, ->>, "\sim\ "] & \rO,
	\end{tikzcd}
\end{equation}
where \(\rO^{\ac}\) is the Koszul dual cooperad associated to a chosen presentation of \(\rO\), \(\Omega\) is the cobar construction, and the morphism \eqref{eq:Omega Oac->O} is induced by the canonical \textit{twisting morphism} \(\rO^{\ac} \to \rO\).

\subsubsection{Twisting morphisms}

A \defn{twisting morphism} between a cooperad \(\rC\) and a operad \(\rO\) is a degree \(-1\) map of \(\Sy\)-modules \(\tau \colon \rC \to \rO\) satisfying the Maurer--Cartan equation in the convolution dg Lie algebra \(\Hom_{\Sy}(\rC,\rO)\):
\[
\partial_{\rO} \circ \tau + \tau \circ \partial_{\rC} + \tau \star \tau = 0.
\]

\begin{lemma}\label{lemma:KDdgInhomo}
	Let \(\big(E_\bullet, R_\bullet\big)\) be dg quadratic-linear data satisfying the quadratic-linear conditions.
	Then, the canonical \(\Sy\)-module map
	\[
	\kappa \colon \mathcal{C}\big(sE_\bullet, s^2 qR_\bullet\big) \twoheadrightarrow sE_\bullet
	\xrightarrow{s^{-1}} E_\bullet \to \mathcal{P}\big(E_\bullet, R_\bullet\big)
	\]
	is a twisting morphism.
	Additionally, \(\kappa\) induces a twisting morphism
	\[
	\bar{\kappa} \colon \big(H_\bullet(\rO^{\ac}, d_1), \bar{d}_\psi\big) \to \big(H_\bullet(\rO, \partial), 0\big).
	\]
	where \(\bar{d}_\psi\) is the differential in the \(d_1\)-homology of \(\rO^\ac\) induced by \(d_\psi\).
\end{lemma}

\begin{proof*}
%	This extends the case of quadratic-linear data with trivial differential treated in \cite[Section~7.8.4]{LodayVallette12}, but the arguments are similar.
	We have to show that the assignment \(\kappa\) satisfies the Maurer--Cartan equation
	\[
	\dif \kappa + \kappa\,(d_1 + d_\psi) + \kappa \star \kappa = 0.
	\]
	By definition of the various maps, the left-hand side of this equation vanishes on elements of weight \(0\) and of weight greater or equal to \(3\).
	For elements \(sE_\bullet\) of weight one, the only possibly non-trivial terms are the first two, which are equal to
	\[
	\partial s^{-1} + s^{-1}(-\partial) = 0.
	\]
	For elements of \(s^2 qR_\bullet\) that are also in \(s^2 R_\bullet\), that is homogeneous quadratic relations, their image under \(\kappa \star \kappa\) vanishes in \(\rO\) since they are initial homogeneous quadratic relations there.
	Their image under \(\kappa d_\psi\) vanishes by definition of \(\psi\), and their image under \(\dif \kappa + \kappa d_1\) vanishes for weight reasons.

	For elements \(s^2 \rho\) of \(s^2 qR_\bullet\) that are not in \(s^2 R_\bullet\), that is inhomogeneous quadratic relations, there exists a non-trivial \(\varepsilon \in E_\bullet\) such that \(\rho + \varepsilon \in R_\bullet\).
	In this case, the image of \(s^2 \rho\) under \(\dif \kappa + \kappa d_1\) still vanishes for weight reasons.
	The remaining non-trivial part of the Maurer--Cartan equation is equal to
	\[
	\kappa\big(d_\psi(s^2\rho)\big) + (\kappa \star \kappa)(s^2\rho)
	=
	\kappa(-s\varepsilon) + \varepsilon
	=
	-\varepsilon + \varepsilon
	=
	0
	\]
	in \(\rO\).

	Finally, since the internal codifferential \(d_1\) preserves the weight grading, the map \(\bar{\kappa}\) is well defined and equal to
	\[
	\bar{\kappa} \colon H_\bullet\big(\rO^{\ac}, d_1\big) \twoheadrightarrow H_\bullet\big(s E_\bullet,-\partial\big)
	\xrightarrow{s^{-1}} H_\bullet\big(E_\bullet,\partial\big) \to H_\bullet\big(\rO, \partial\big).
	\]
	The Maurer--Cartan equation on the chain level satisfied by \(\kappa\) implies the Maurer--Cartan equation on homology satisfied by \(\bar{\kappa}\).
\end{proof*}

%\begin{lemma}
%	It induces a twisting morphism
%	\[
%	\bar{\kappa} \colon H_\bullet\big(\rO^{\ac}, d_1\big) \to H_\bullet\big(\rO,\partial\big)
%	\]
%	from the homology of the Koszul dual cooperad equipped with the induced codifferential \(\bar{d}_\psi\) to the homology operad.
%\end{lemma}

\subsubsection{Koszul property}

A twisting morphism \(\tau \colon \rC \to \rO\) is called \defn{Koszul} when its associated operad morphism \(\Omega\rC \to \rO\), which is always a fibration, is a weak equivalence.
This morphism is the unique operad map whose restriction to the generators \(s^{-1}\overline{\rC} \subset \Omega\rC\) sends \(s^{-1}c\) to \(\tau(c)\) for each \(c \in \overline{\rC}\).

\begin{definition}
	An operad \(\rO = \mathcal{P}\big(E_\bullet, R_\bullet\big)\) given by dg quadratic-linear data satisfying the
	quadratic-linear conditions is called \defn{Koszul} when \(\bar{\kappa}\) is a Koszul morphism.
\end{definition}

\begin{theorem}\label{t:koszul_main}
	The cobar construction \(\Omega\rO^\ac\) of the Koszul dual cooperad of an Koszul operad \(\rO = \mathcal{P}\big(E_\bullet, R_\bullet\big)\) is cofibrant and the canonical projection
	\[
	\begin{tikzcd}[column sep=0.7cm]
		\rO_\infty \defeq \Omega \rO^{\ac}
		\arrow[r, ->>, "\sim"]  & \rO
	\end{tikzcd}
	\]
	is an acyclic fibration.
\end{theorem}

\begin{proof*}
	This twisting morphism \(\kappa\) of \cref{lemma:KDdgInhomo} induces a morphism of operads
	\[
	g_\kappa \colon \rO_\infty = \Omega \rO^{\ac} \twoheadrightarrow \rO,
	\]
	which is surjective since it reaches all the generators of \(\rO\), so it is a fibration.
	Let us recall that the cobar construction of the Koszul dual cooperad is given by
	\[
	\Omega \rO^{\ac} =
	\Big(
	\mathcal{T}\big(s^{-1}\overline{\mathcal{C}}\big(sE_\bullet, s^2 qR_\bullet\big)\big),
	d_1 + d_\psi + d_2
	\Big),
	\]
	where we use the same notation for the differentials induced by \(d_1\) and \(d_\psi\), by a slight abuse of notation.
	In order to prove that the map \(g_\kappa\) is a quasi-isomorphism, we consider, on the left-hand side, the increasing filtration
	\[
	\mathcal{F}_k \Omega \rO^{\ac} \defeq \bigoplus_{l \leqslant k} \Omega_l \rO^{\ac},
	\qquad
	0 = \mathcal{F}_{-1} \subset \mathcal{F}_0 \subset \mathcal{F}_1 \subset \mathcal{F}_2 \subset \cdots \subset \mathcal{F}_k \subset \mathcal{F}_{k+1} \subset \cdots \subset \Omega\rO^{\ac}
	\]
	given by the syzygy degree, which is defined for every element of \(\Omega\rO^{\ac}\) by the sum over the vertices of the weight of the labeling elements minus \(1\), see \cite[Section~7.3]{LodayVallette12}.
	The differential \(d_1\) preserves this filtration and the two differentials \(d_\psi\) and \(d_2\) lower it by \(1\).
	On the right-hand side, we consider the trivial filtration
	\[
	0 = \mathcal{G}_{-1} \subset \mathcal{G}_0 = \rO = \mathcal{G}_1 = \cdots = \mathcal{G}_k = \cdots = \rO.
	\]
	The assignment \(g_\kappa\) obviously preserves the respective filtrations.
	On the left-hand side, the first page of the associated spectral sequence is isomorphic to
	\[
	\big(E^0 \Omega \rO^{\ac}, d^0\big) \cong \Big(\mathcal{T}\big(s^{-1}\overline{\mathcal{C}}\big(sE_\bullet,
	s^2 qR_\bullet\big)\big), d_1\Big).
	\]
	The operadic Künneth theorem \cite[Proposition~6.2.3]{LodayVallette12} implies that
	the second page of this spectral sequence is isomorphic to the cobar construction of the homology of the Koszul dual cooperad equipped with the codifferential \(\bar{d}_\psi\):
	\[
	\big(E^1 \Omega \rO^{\ac}, d^1\big) \cong \Omega H_\bullet\big(\rO^\ac, d_1\big).
	\]
	On the right-hand side, the first page of the associated spectral sequence is isomorphic to
	\[
	\big(E^0 \rO, d^0\big) \cong \big(\rO, \dif\big),
	\]
	so its second page is isomorphic to the homology operad equipped with trivial differential:
	\[
	\big(E^1 \rO, d^1\big) \cong \big(H_\bullet(\rO, \partial), 0\big).
	\]
	The induced map
	\[
	\begin{tikzcd}
		E^1 g_\kappa = g_{\bar\kappa} \colon \Omega H_\bullet\big(\rO^\ac, d_1\big)
		\arrow[r, "\sim"]  & H_\bullet(\rO, \partial)
	\end{tikzcd}
	\]
	is the quasi-isomorphism of operads corresponding to the Koszul morphism
	\(\bar{\kappa} \colon H_\bullet\big(\rO^{\ac}, d_1\big) \to H_\bullet\big(\rO,\partial\big)\).
	Since these two filtrations are bounded below and exhaustive, their convergence shows that the morphism \(g_\kappa\) is a quasi-isomorphism by \cite[Theorem~5.2.12]{WeibelBook}.

	\medskip

	This Koszul type resolution
	\[
	\begin{tikzcd}
		\rO_\infty = \Omega \rO^{\ac}
		=
		\Big(
		\mathcal{T}\big(s^{-1}\overline{\mathcal{C}}\big(sE_\bullet, s^2 qR_\bullet\big)\big), d_1 + d_\psi + d_2
		\Big)
		\arrow[r, ->>, "\sim"]  & \rO
	\end{tikzcd}
	\]
	is quasi-free on a space of generators non-negatively graded by the syzygy degree, so it is cofibrant.
\end{proof*}