\subsection{Model categorical relationship}\label{ss:homotopy_fibre}

We start from the canonical diagram of dg quadratic-linear data
\[
\begin{tikzcd}[column sep=small]
	(V, S) \rar & (E_\bullet, R_\bullet) \rar & (E, R)
\end{tikzcd}
\]
defining the diagram \(\Com \to \cBV \to \BV\) studied in \cref{thm:Homology}.
Passing to the associated Koszul dual cooperads we obtain a diagram of cooperads
\begin{equation*}\label{eq:diagram_cooperads}
	\begin{tikzcd}[column sep=small]
		\Com^{\ac} \rar & \cBV^{\ac} \rar & \BV^{\ac}.
	\end{tikzcd}
\end{equation*}
Applying the cobar construction to it yields a diagram of operads
\[
\begin{tikzcd}[column sep=small]
	\Omega\Com^{\ac} \rar & \Omega\cBV^{\ac} \rar & \Omega\BV^{\ac}.
\end{tikzcd}
\]

\begin{theorem}\label{prop:FactAcyCofFib}
	The diagram
	\[
	\begin{tikzcd}[column sep=20pt]
		\Com_\infty \arrow[r, >->, "\sim"] & \cBV_\infty \arrow[r,->>] & \BV_{\!\infty}
	\end{tikzcd}
	\]
	is made up of an acyclic cofibration followed by a fibration.
\end{theorem}

\begin{proof*}
	The above diagram of cooperads extends to a short exact sequence.
	\[
	\begin{tikzcd}[column sep=small]
		0 \rar & \Com^{\ac}  \rar & \cBV^{\ac} \rar & \BV^{\ac} \rar & 0
	\end{tikzcd}
	\]
	Forgetting the differentials, it corresponds to
	\[
	\begin{tikzcd}[column sep=small]
		\Com^{\ac} \rar & \Com^\ac \oplus \rM^* \oplus s^{-1}\rM^* \rar & \Com^\ac \oplus \rM^*
	\end{tikzcd}
	\]
	by \cref{thm:FormcBVac}, with the canonical inclusion and projection defined by the direct summands.
	Since the cobar construction is a left adjoint, we get the exact sequence
	\[
	\begin{tikzcd}[column sep=small]
		\Com_\infty \rar & \cBV_\infty \rar & \BV_\infty \rar & 0
	\end{tikzcd}
	\]
	and this implies that the map \(\cBV_\infty \to \BV_\infty\) is a fibration.
	Let us now focus on the first map.
	Forgetting the differentials, it is the inclusion into the coproduct
	\[
	\Com_\infty \to \Com_\infty \vee \mathcal{T}(s^{-1}\rM^* \oplus s^{-2}\rM^*).
	\]
	Both operads \(\Com_\infty\) and \(\cBV_\infty\) are non-negatively graded.
	The degree \(0\) summand of the generating space \(s^{-1}\rM^* \oplus s^{-2}\rM^*\) is spanned by \(s^{-1}(s \c) \cong \c\) and \(s^{-1}(s \square) \cong \square\), whose differential vanishes.
	Defining the increasing and exhaustive filtration
	\[
	\mathrm{S}_k \coloneqq \bigoplus_{l=0}^k (s^{-1}\rM^* \oplus s^{-2}\rM^*)_l,
	\]
	we obtain an increasing and exhaustive filtration \(\Com_\infty \vee \mathcal{T}(\mathrm{S}_k)\) of the operad \(\cBV_\infty\) satisfying
	\[
	(d_1 + d_\psi + d_2)(S_k) \subset \Com_\infty \vee \mathcal{T}(\mathrm{S}_{k-1})
	\]
	for all \(k \geqslant  0\).
	This proves that the map \(\Com_\infty \to \cBV_\infty\) is a cofibration, which is acyclic by the arguments of \cref{t:main}.
\end{proof*}

A straightforward consequence of this theorem is that the kernel of the map \(\cBV_{\!\infty} \to \BV_{\!\infty}\) is a model for the \textit{homotopy fibre} of the map \(\Com_\infty \to \BV_{\!\infty}\).
By \cref{thm:FormcBVac}, this kernel is the ideal of \(\cBV_{\!\infty}\) generated by \(s^{-2}\rM^*\).
It is explicitly given by the linear span of trees with vertices labelled by elements of \(s^{-1}\overline{\cBV}^{\ac}\) such that at least one vertex is labelled by an element of \(s^{-2}\rM^*\).

%\begin{corollary}
%	The kernel of the map \(\cBV_{\!\infty} \to \BV_{\!\infty}\) is a model for the homotopy fibre of the map \(\Com_\infty \to \BV_{\!\infty}\).
%\end{corollary}
%
%\begin{proof*}
%	In the projective type model category of operads of \cite{Hinich97}, all objects are fibrant.
%	Therefore the homotopy fibre of \(\Com_\infty \to \BV_\infty\) is given by the fibre of a fibrant replacement of it.
%	By \cref{prop:FactAcyCofFib}, this replacement is provided by the kernel of the morphism \(\cBV_\infty \to \BV_\infty\).
%\end{proof*}
%
%This kernel is the ideal of \(\cBV_{\!\infty}\) generated by \(s^{-2}\rM^*\), by \cref{thm:FormcBVac}.
%It is explicitly given by the linear span of trees with vertices labelled by elements of \(s^{-1}\overline{\cBV}^{\ac}\) such that at least one vertex is labelled by an element of \(s^{-2}\rM^*\).

%\begin{remark}
%	At first sight, one might think that, since the operad \(\cBV\) is quasi-isomorphic to the operad \(\Com\), the cofibrant replacement \(\Com_\infty\) suffices to resolve \(\cBV\).
%	There is indeed a canonical quasi-isomorphism \(\Com_\infty \xrightarrow{\sim} \cBV\), but it is not an acyclic fibration.
%	Any morphism of operads \(\Com_\infty \to \cBV\) cannot be surjective for degree reasons.
%	This explains why the operad \(\cBV_\infty\) is required in order to obtain a cofibrant replacement of the operad \(\cBV\).
%\end{remark}
