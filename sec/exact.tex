\subsection{Exact and coexact \(\BV\)-structures}\label{ss:exact}

This subsection is not part of the logical dependency structure of the paper and can be safely skipped on a first reading.
Its purpose is to explain our choice of terminology by comparing the operad \(\cBV\) to the operad \(\eBV\) controlling \textit{exact} BV-algebras.

\subsubsection{Exact \(\BV\)-algebras and Poisson manifolds}

Recall that in a (dg) \(\BV\)-algebra \((A, d, \cdot, \triangle)\), the operators \(d\) and \(\triangle\) commute; that is, \([d, \triangle] = 0\).
This means that \(\triangle\) is \emph{closed} in \(\End(A)\).
The operator \(\triangle\) is said to be \emph{exact} if there exists a linear map \(\nabla \colon A \to A\) such that \(\triangle = [d, \nabla]\).
As usual, exactness implies closedness.
Therefore, the following notion, due to Guan--Muro \cite[Definition~4.5]{GuanMuro23} and of relevance to Poisson geometry, provides a refinement of the concept of \(\BV\)-algebra.

\begin{definition}
	An \defn{exact \(\BV\)-algebra} is a tuple \((A, d, \cdot, \nabla)\), where \((A, d, \cdot)\) is a dg commutative algebra and \(\nabla \colon A \to A\) is a linear map of degree \(-2\) and order at most \(2\), satisfying
	\begin{equation}\label{Eq:Nabla}
		[\nabla,[\nabla,d]] = 0.
	\end{equation}
\end{definition}

The following proposition, proven straighforwardly using Cartan calculus, shows that Poisson manifolds canonically define exact \(\BV\)-algebras.

\begin{proposition}[{\cite[Section~3]{Koszul85}}]\label{prop:eBVPoisson}
	Let \(M\) be a manifold with a section \(\pi \in \Gamma(\Lambda^2 TM)\) such that \([\pi, \pi] = 0\).
	Then the de Rham complex \(\Omega^{\bullet}(M)\) admits an exact \(\BV\)-algebra structure
	\[
	\big(\Omega^{\bullet}(M), d_{dR}, \wedge, \iota_\pi\big),
	\]
	where \(\iota_\pi\) denotes contraction with \(\pi\).
\end{proposition}

\medskip\noindent
Exact \(\BV\)-algebras are algebras over the (dg) operad
\[
\eBV \defeq \dfrac{\BV \vee \mathcal{T}(\nabla)}
{\big([\triangle, \nabla], \
	\nabla \ \text{second-order}\big)},
\]
whose differential is determined by
\begin{equation}\label{eq:exactness}
	\dif \, \nabla \defeq \triangle.
\end{equation}
In contrast to \(\BV\)-algebras, the operator \(\triangle\) is always \(0\) in the homology of \(\eBV\)-algebras.
As we will see next, this is also the case in the homology of \(\cBV\)-algebras that are not \(\BV\)-algebras, not because \(\triangle\) is exact, but because it is not closed.

\subsubsection{Coexact \(\BV\)-algebras and pseudo-Riemannian manifolds}

Recall that the operad \(\cBV\) is given by
\[
\dfrac{\BV \vee \mathcal{T}(\square)}
{\big([\triangle, \square], \ \square \ \text{second-order}\big)},
\]
with differential determined by
\[
\dif \, \triangle = \square.
\]
Here the generator \(\triangle\) is \emph{coexact}, explicitly \(\dif^*(\square) = \triangle\), where \(\dif^*\) is the adjoint of \(\dif\) with respect to the canonical inner product.\footnote{
	Explicitly, this inner product is defined by the canonical basis of \(\cBV(1)\), which in degree \(\bars{\triangle}\) is \(\set{\mathrm u_{m} =\triangle \circ \square^{\circ m} \mid m \geqslant 0}\).
	We can verify the claim \(\dif^*(\square) = \triangle\) by writing \(\dif^*(\square) = \sum_{m} \alpha_{m} \cdot \mathrm u_{m}\) with \(\alpha_{m} = \angles{\mathrm u_{m}, \dif^*(\square)} = \angles{\dif(\mathrm u_{m}), \square} = \angles{\square^{\circ m+1}, \square}\), and concluding that
	\[
	\alpha_{m} =
	\begin{cases}
		1 & m = 0,\\
		0 & \text{otherwise}.
	\end{cases}
	\]}
This motivates the terminology \defn{coexact \(\BV\)-algebras} and the notation \(\cBV\) for the operad controlling them.
We remind the reader that \(\BV\)-algebras are precisely coexact dg \(\BV\)-algebras where the operator corresponding to \(\square\), the obstruction \(n\), vanishes.

We have the following analogue of \cref{prop:eBVPoisson} stating that pseudo-Riemannian manifolds canonically define coexact \(\BV\)-algebras.

\begin{proposition}\label{prop:deRhamCoexact}
	Let \(M\) be a manifold equipped with a nowhere-degenerate section \(g \in \Gamma(S^2 T^*M)\).
	Then the de Rham complex \(\Omega^{\bullet}(M)\) admits a canonical coexact \(\BV\)-algebra structure
	\[
	\big(\Omega^{\bullet}(M), d_{dR}, \wedge, d^\star\big),
	\]
	where \(d^\star\) is the Hodge codifferential.
\end{proposition}

The parallel, or duality, observed here between Poisson and pseudo-Riemannian manifolds is best understood in the language of supergeometry, although we do not develop this perspective further.

\subsubsection{Relationship between \(\BV\)-type operads}

We conclude this subsection making explicit the relationship between the operads we have discussed so far.

\begin{theorem}\label{thm:Homology}
	The canonical maps induced by the identification of generators define the commutative diagram
	\[
	\begin{tikzcd}[row sep=small]
		\cBV \arrow[r, two heads] \arrow[rr, bend left=35, "\sim"] &
		\BV \arrow[r, hook] &
		\eBV \\
		& \Com \arrow[ul, hook', "\sim"', bend left] \arrow[ur, hook, "\sim", bend right] &
	\end{tikzcd}
	\]
	where \(\hookrightarrow\) denotes an injection, \(\twoheadrightarrow\) a surjection, and \(\xrightarrow{\sim}\) a quasi-isomorphism.
\end{theorem}

\begin{proof}
	The inclusion \(\Com \hookrightarrow \eBV\) is a quasi-isomorphism by \cite[Theorem~1.3]{GuanMuro23}.
	To analyze \(\Com \hookrightarrow \cBV\) we adapt the method of \cite[Section~2]{DrummondColeVallette13}, and we note that this argument also yields an alternative proof of the acyclicity of \(\Com \hookrightarrow \eBV\).

	The operad \(\cBV\) carries a weight grading defined by declaring that the total number of occurrences of \(\triangle\) and \(\square\) is the weight.
	Both the relations and the differential are homogeneous for this grading.
	Let \(\cBV^{[k]}\) denote the weight-\(k\) summand.
	Then \(\cBV^{[0]} \cong \Com\).

	Consider the assignment
	\[
	\cdot \mapsto 0,
	\qquad
	\triangle \mapsto 0,
	\qquad
	\square \mapsto \triangle.
	\]
	This is a morphism of quadratic data and therefore extends uniquely to a derivation \(h \colon \cBV \to \cBV\) preserving the weight.
	Define a degree \(1\) morphism of \(\Sy\)-modules \(H \colon \cBV \to \cBV\) by
	\[
	H \mid_{\cBV^{[0]}} \coloneq 0,
	\qquad
	H \mid_{\cBV^{[k]}} \coloneq \frac{1}{k} \, h \mid_{\cBV^{[k]}}
	\quad (k \ge 1).
	\]

	Let
	\[
	\begin{tikzcd}[column sep=small]
		i \colon \Com \arrow[r, hook] & \cBV
	\end{tikzcd}
	\qquad\text{and}\qquad
	\begin{tikzcd}[column sep=small]
		p \colon \cBV \arrow[r, two heads] & \Com
	\end{tikzcd}
	\]
	denote the canonical inclusion and projection.
	Then \((i,p,H)\) is a deformation retract in the category of dg \(\Sy\)-modules, namely
	\[
	H \dif + \dif H
	=
	\id_{\cBV} - i p.
	\]

	The operator \(H \dif + \dif H\) is determined by its effect on generators:
	\[
	\cdot \mapsto 0,
	\qquad
	\triangle \mapsto \triangle,
	\qquad
	\square \mapsto \square.
	\]
	Thus \(i\) is a quasi-isomorphism.
	Finally, the 2-out-of-3 property of weak equivalences concludes the proof.
\end{proof}

\begin{remark}
	\cref{thm:Homology} shows that, for a pseudo-Riemannian manifold, the coexact BV-algebra structure on differential forms yields no homotopical invariants on de~Rham cohomology beyond the Massey products induced by the wedge product.
	The corresponding statement for the exact BV-algebra structure in the Poisson case was made in \cite{GuanMuro23}.
\end{remark}