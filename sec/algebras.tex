% !TEX root = ../ck2.tex

\section{A model for homotopy \(\cBV\)-algebras}\label{sec:algebras}

In this section we introduce \(\cBV_{\!\infty}\)-algebras in concrete terms.
We begin by recalling the conventions underlying our constructions and review \(\Com\)-algebras, \(\cBV\)-algebras, and \(\BV\)-algebras.
We then introduce \(\cBV_{\!\infty}\)-algebras via generating maps, show how every \(\rC_\infty\)-algebra extends canonically to such a structure, and define obstruction maps measuring the failure of a \(\cBV_{\!\infty}\)-algebra to descend to \(\BV_{\!\infty}\)-algebra.
A weight filtration is introduced to organize the hierarchy of homotopies present in \(\cBV_{\!\infty}\)-algebras and define \(\cBV_{\!(n)}\)-algebras for \(n \in \N\).
Finally, we illustrate the utility of this framework with the kinematic algebra of Yang--Mills theory \cite{Bonezzi2023GaugeInvariantDoubleCopyQuartic, Bonezzi2024WeaklyConstrainedDoubleField}.

\subsection{Conventions}

\subsubsection{The underlying category}\label{ss:dgVec}

We work throughout in the closed symmetric monoidal category of cochain complexes over a field \(\KK\) of characteristic \(0\).
We spell this out for the convenience of the reader.

\medskip\noindent
A \textit{cochain complex} is a pair \((V, d)\) consisting of a graded vector space \(V = \bigoplus_{n \in \Z} V^n\) and a \textit{differential}, that is, a linear map \(d \colon V \to V\) such that \(d(V^n) \subseteq V^{n+1}\) for all \(n \in \Z\) and \(d \circ d = 0\).

\medskip\noindent
Saying that the category is \emph{monoidal} means that it is equipped with a product and a unit object satisfying the usual associativity and unitality constraints.
In this case, we have the tensor product
\[
(V \otimes W)^n \;=\; \bigoplus_{p+q=n} V^p \otimes_\KK W^q,
\]
with differential
\[
d_{V \otimes W}(v \otimes w) \;\defeq\; d(v) \otimes w \;+\; (-1)^{|v|} v \otimes d_W(w),
\]
and the unit object being \(\KK\) concentrated in degree \(0\) with trivial differential.

\medskip\noindent
Saying that the category is \emph{closed} means that we have an \textit{internal Hom}: for any cochain complex \(V,W\), the collection of linear maps from \(V\) to \(W\) itself forms a cochain complex
\[
\Hom(V,W) \;=\; \bigoplus_{k \in \Z} \Hom^k(V,W),
\]
where \(\Hom^k(V,W)\) consists of all linear maps \(f \colon V \to W\) of \textit{degree \(k\)}, i.e.\ satisfying \(f(V^n) \subseteq W^{n+k}\) for all \(n\) in \(\Z\).
The differential on \(\Hom(V,W)\) is given by
\[
d_{\Hom}(f) \;\defeq\; d_W \circ f \;-\; (-1)^{|f|} f \circ d .
\]
This internal Hom gives rise to a canonical \textit{dg adjunction}:
\[
\Hom(U \otimes V, W) \;\cong\; \Hom\big(U,\, \Hom(V,W)\big)
\]
for any \(U,V,W\).

\medskip\noindent
Saying that this category is \emph{symmetric} asserts the existence of the isomorphism
\begin{equation}\label{eq:symmetry_isomorphism}
	\begin{tikzcd}[row sep=0, column sep=small]
		V \otimes W \arrow[r, "\tau_{V,W}", shift left=0.5ex] & W \otimes V \\
		v \otimes w \arrow[mapsto, r] & (-1)^{|v||w|}\, w \otimes v,
	\end{tikzcd}
\end{equation}
for all homogeneous elements \(v \in V\) and \(w \in W\).

\subsubsection{Symmetric group action}

The \textit{symmetry isomorphism} \eqref{eq:symmetry_isomorphism} defines a left action of the symmetric group \(\Sy_n\) on \(V^{\otimes n}\) by
\[
\sigma (v_1 \otimes \dotsb \otimes v_n)
\defeq
\pm\, v_{\sigma^{-1}(1)} \otimes \dotsb \otimes v_{\sigma^{-1}(n)},
\]
for all \(\sigma \in \Sy_n\) and homogeneous \(v_i \in V\),
where the sign is determined by the \emph{Koszul convention}.
%\footnote{Exchanging two homogeneous objects of odd degree introduces a minus sign, while all other transpositions leave the sign unchanged.}.

This induces a natural right \(\Sy_n\)-action on \(\Hom(V^{\otimes n},V)\) given by precomposition.
Explicitly, for \(f \in \Hom(V^{\otimes n},V)\) and \(\sigma \in \Sy_n\)
\[
(f \circ \sigma^{-1})(v_1 \otimes \dotsb \otimes v_n)
\defeq
\pm\, f \big(v_{\sigma(1)} \otimes \dotsb \otimes v_{\sigma(n)}\big).
\]

\subsubsection{Partial compositions and the insertion bracket}

Given \(f \in \Hom(V^{\ot n}, V)\), \(g \in \Hom(V^{\ot m}, V)\), and \(1 \leqslant i \leqslant n\), the \defn{partial composition}
\[
f \circ_i g \;\in\; \Hom\!\big(V^{\ot (n+m-1)}, V\big)
\]
is defined on homogeneous inputs \(v_1 \ot\dotsb\ot v_{n+m-1}\) by
\[
\begin{split}
	(-1)^{|g| \cdot (|v_1| + \dotsb + |v_{i-1}|)}
	f\big(v_1 \ot\dotsb\ot v_{i-1} \ot g(v_i \ot\dotsb\ot v_{i+m-1})
	\ot v_{i+m} \ot\dotsb\ot v_{n+m-1}\big).
\end{split}
\]

\medskip\noindent
From these partial compositions we define the \defn{pre-Lie product}
\begin{equation}\label{eq:preLie_product}
	f \star g \;\defeq\; \sum_{i=1}^{n} f \circ_i g ,
\end{equation}
and its antisymmetrization, the \defn{insertion bracket},
\[
[f,g] \;\defeq\; f \star g \;-\; (-1)^{|f||g|}\, g \star f.
\]
This bracket makes
\[
\End_V \;\defeq\; \prod_{n \geqslant 1} \Hom(V^{\ot n},V)
\]
into a \emph{dg Lie algebra} extending the (graded) commutator Lie algebra on \(\Hom(V,V)\).

\subsection{Strict algebras}

A differential graded commutative algebra, or more compactly a \defn{\(\Com\)-algebra}, is a tuple \((A,d,m)\), where \((A,d)\) is a cochain complex and \(m \colon A \ot A \to A\) is a (graded) symmetric bilinear map of degree \(0\), i.e. \(m \circ \big(\id - (12)\big) = 0\), such that:
\[
m \circ_1 m - m \circ_2 m = 0, \qquad
[d, m] = 0.
\]
The first identity states the associativity of \(m\) and the second that \(d\) is a \textit{derivation} of \(m\).

\begin{definition}\label{def:cbv_algebra}
	A \defn{\(\cBV\)-algebra} (\emph{coexact Batalin--Vilkovisky algebra}) is a tuple \((A,d,m,\triangle)\), where \((A,d,m)\) is a \(\Com\)-algebra and \(\triangle \in \Hom(A, A)\) is a degree \(-1\) map satisfying \(\triangle \circ \triangle = 0\) and the \textit{second-order relation}:
	\begin{equation}\label{eq:second-order}
		\begin{split}
			\triangle(abc)
			&= \triangle(ab)c
			+ (-1)^{|a|}a\,\triangle(bc)
			+ (-1)^{(|a|+1)|b|}b\,\triangle(ac) \\[4pt]
			&\quad
			- \triangle(a)bc
			- (-1)^{|a|}a\,\triangle(b)c
			- (-1)^{|a|+|b|}ab\,\triangle(c),
		\end{split}
	\end{equation}
	where \(m(a \otimes b)\) is denoted \(ab\).
\end{definition}

\medskip\noindent A (differential graded) \defn{\(\BV\)-algebra}, short for (dg) \textit{Batalin--Vilkovisky algebra}, is a \(\cBV\)-algebra  \((A, d, m, \triangle)\) where the \defn{obstruction}
\[
n \defeq [d, \triangle]
\]
is identically \(0\).

\begin{remark}
	Both \(\cBV\) and \(\BV\) algebras may also be defined in the absence of a differential, in which case the two notions coincide.
	Throughout, we restrict attention to their differential graded versions.
\end{remark}

\begin{example}
	For a smooth manifold \(M\), the \(\Com\)-algebra \((\Omega(M), d, \wedge)\) of differential forms extends to a \(\cBV\)-algebra once a non-degenerate metric is chosen, with the Hodge codifferential \(d^\star\) playing the role of \(\triangle\).
	In the Lorentzian case, the obstruction \([d, d^\star]\) is the d’Alembertian, typically denoted \(\square\).
	This is the origin of the alternative terminology \(\BV^\square\)-algebras for \(\cBV\)-algebras \cite{Reiterer2020HomotopyBVYMCK}.
\end{example}

In a \(\cBV\)-algebra, the failure of \(\triangle\) to be a derivation of the product \(\m\) defines a degree \(-1\) symmetric bracket \(b \defeq [\triangle,\m]\) satisfying the \defn{Jacobi relation}
\[
(b \,\circ_1 b) \circ \big(\id + (123) + (132)\big) = 0
\]
and the \defn{Leibniz relation}
\[
b \,\circ_1 m = m \,\circ_1 b + (m \,\circ_2 b) \circ (213).
\]
The latter is equivalent to the second-order relation~\eqref{eq:second-order}.
Thus \((A,d,\m,b)\) is almost a \textit{Gerstenhaber algebra} except that \([d,b] = [d,[\triangle,\m]] = [[d,\triangle],\m] = [n,\m]\) need not vanish if the obstruction \(n \neq 0\).

\subsection{\(\cBV_{\!\infty}\)-algebras}\label{ss:generating_maps}

Recall that a \defn{\(\rC_\infty\)-algebra} is a graded vector space \(A\) together with maps
\[
m_n \colon A^{\ot n} \to A, \qquad n \geqslant  1,
\]
of degree \(2-n\) with the following symmetries holding for every \(j\):
\[
\sum_{\mathclap{\sigma \in \Sh(j,\, n - j)}} \ \sign(\sigma)\,
m_n  \! \circ \sigma =  0.
\]
They satisfy the relation:
\[
\sum_{r+s+t = n} (-1)^{r + st}\, m_{r+1+t} \circ (\id^{\ot r} \ot \, m_s \ot \id^{\ot t}) = 0
\]
for any \(n \geqslant 1\).
If \(m_k = 0\) for all \(k > n+1\) we say that this is a \defn{\(\rC_{(n)}\)-algebras}.

\medskip

Let us fix a graded vector space \(A\).
A set of \defn{generating maps} on \(A\) consists of a collection of linear maps
\begin{align*}
	m_{p_1,\dots,p_k}^t &\colon
	A^{\ot p_1} \ot\dotsb\ot A^{\ot p_k}
	\longrightarrow A,
\end{align*}
for each \(	t \geqslant  0, \  k \geqslant  1, \  p_1, \dots, p_k \geqslant  1\)\,, of degree
\[
\bars{m_{p_1, \dots, p_k}^t}  =  3 - 2t - p_1 - \cdots - p_k - k,
\]
with the following \defn{block} and \defn{shuffle symmetries}:

\medskip\noindent(1)
For any permutation \(\sigma \in \sym_k\):
\[
m_{p_{\sigma^{-1}(1)},\dots,p_{\sigma^{-1}(k)}}^t \!= m_{p_1, \dots, p_k}^t \! \circ \overline\sigma,
\]
where \(\overline\sigma\) is the image of \(\sigma\) via the block inclusion \(\sym_k \to \sym_{p_1+\dots+p_k}\).

\medskip\noindent(2)
For any \(i \in \set{1,\dots,k}\) and \(j \in \set{1, \dots, p_i-1}\):
\[
\sum_{\mathclap{\ \sigma \in \Sh(j,\, p_i - j)}} \ \sign(\sigma) \;
m_{p_1,\dots,p_k}^t  \! \circ \underline\sigma =  0,
\]
where \(\underline\sigma\) is the image of \(\sigma\) in via the inclusion \(\sym_{p_i} \to \sym_{p_1+\dots+p_k}\) induced by the \(i^\th\)-block.

\medskip In \cref{label} we will prove that the following definition coincides with that of an algebra over a cofibrant resolution of the operad controlling \(\cBV\)-algebras.

\begin{definition}\label{def:cBV-algebra}
	A \defn{\(\cBV_{\!\infty}\)-algebra} is a graded vector space \(A\) together with a set of generating maps \(\set{m_{p_1, \dots, p_k}^t}\) such that \((A, m^0_1, m^0_2, m^0_3, \dots )\) is a \(\rC_\infty\)-algebra.
\end{definition}

\noindent A straightforward consequence of this characterization is the following.

\begin{theorem}\label{t:extension from C_infty}
	Any \(\rC_\infty\)-algebra \(\big(A, m_1, m_2, m_3, \dots\big)\) can be extended to a \(\cBV_{\!\infty}\)-algebra by completing \(\set{m^0_i \defeq m_n}_{n \geqslant 1}\) to a set of generating maps.
\end{theorem}

\noindent A canonical, although not very interesting choice, is to set all the additional maps to \(0\).

\subsection{Obstruction maps and \(\BV_\infty\)-algebras}\label{ss:obstruction_maps}

\subsubsection{Straight shuffles}

To understand explicitly the relationship between \(\cBV_{\!\infty}\)-algebras and \(\BV_{\!\infty}\)-algebras, we will need the combinatorial notion of \textit{straight shuffle}, which we illustrate with the following example:
\begin{equation}\label{ex:straight_shuffle}
	1\;\underline{2}\;3\;4\;|\;\underline{5\;6}\;7\;8\;|\;9\;\underline{10\;11}\;|\;\underline{12\;13}
	\ \longrightarrow\
	9\;1\;\;\underline{2\;5\;6\;10\;11\;12\;13}\;\;7\;3\;8\;4.
\end{equation}

\medskip\noindent Forgetting all decoration in \eqref{ex:straight_shuffle}, we have the permutation
\[
\sigma =
\left[
\begin{array}{ccccccccccccc}
	1&2&3&4&5&6&7&8&9&10&11&12&13\\
	2&3&11&13&4&5&10&12&1&6&7&8&9
\end{array}
\right]
\in \sym_{13}.
\]
The segmentation induced by the vertical bars in \eqref{ex:straight_shuffle} is encoded in the \(k\)-tuple
\[
\bar p \defeq (4,4,3,2)
\]
with \(k = 4\).
The subdivision of each segment of \eqref{ex:straight_shuffle} induced by the underline is encoded in the three \(k\)-tuples
\[
\bar l = (1,0,1,0), \quad
\bar q = (1,2,2,2), \quad
\bar r = (2,2,0,0).
\]
By definition, the permutation \(\sigma\) is a \defn{ \((\bar l, \bar q, \bar r, \bar p)\)-shuffle} if for all \(1 \leqslant i \leqslant k\),
\[
(p_1+\dots+p_{i-1} + l_i) + [1, q_i ] \xmapsto{\sigma}
(l_1 + \dotsb + l_k) + (q_1 + \dotsb + q_{i-1}) + [1, q_i].
\]
Explicitly,
\[
2 \xmapsto{\sigma} 3, \quad
[5, 6] \xmapsto{\sigma} [4, 5], \quad
[10, 11] \xmapsto{\sigma} [6, 7], \quad
[12, 13] \xmapsto{\sigma} [8, 9].
\]
%Notice that \(\bar p = \bar l + \bar q + \bar r\) and that any other \((\bar l, \bar q, \bar r, \bar p)\)-shuffle is of the form \(\underline\pi \circ \sigma\), where
%\[
%\begin{tikzcd}[row sep=0,column sep=small]
%	\sym_2 \times \sym_4 \rar & \sym_{2+7+4} \\
%	\pi \arrow[r, maps to] & \underline{\pi}.
%\end{tikzcd}
%\]

\medskip

We will be interested in the set  \(\mathrm{StSh}(\bar q, \bar p)\) of all straight shuffles for a fixed pair
\[
\bar q = (q_1, \dots, q_k) \leqslant (p_1, \dots, p_k) = \bar p.
\]

\subsubsection{Straight shuffle extensions}

Let us now introduce some notation for the action of straight shuffles on tensor product elements of a graded vector space.
Using the straight shuffle \(\sigma\) from example \eqref{ex:straight_shuffle}, we have that if \(x = x_1 \ot\dotsb\ot x_{13}\) then
\[
\sigma x =
\pm\, \underbrace{x_9 \ot x_1}_{\sigma x^{(1)}} \ot
\underbrace{x_2 \ot x_5 \ot x_6 \ot x_{10} \ot x_{11} \ot x_{12} \ot x_{13}}_{\sigma x^{(2)}} \ot
\underbrace{x_7 \ot x_3 \ot x_8 \ot x_4}_{\sigma x^{(3)}},
\]
where the sign is given by the permutation of elements under the Koszul sign rule.

\medskip

The \defn{straight shuffle extension} of a map
\[
m_{q_1, \dots, q_k} \colon A^{\ot q_1} \ot\dotsb\ot A^{\ot q_k} \to A
\]
to a map
\[
m_{\frac{p_1,\dots,p_k}{q_1,\dots,q_k}} \colon {A^{\ot p_1}} \ot\dotsb\ot {A^{\ot p_k}} \to A,
\]
where \((q_1, \dots, q_k) \leqslant (p_1, \dots, p_k)\), is defined by
\[
m_{\frac{p_1,\dots,p_k}{q_1,\dots,q_k}}(x)
=
\sum_{\quad \mathclap{\sigma \in \mathrm{StSh}(\bar q, \bar p)}}
%(-1)^{\varepsilon+\varepsilon'+\varepsilon''} \mathrm{sgn}(\sigma)\,
\, \pm \,
\sigma x^{(1)} \ot
m_{q_1,\dots,q_k}\big( \sigma x^{(2)} \big) \ot \sigma x^{(3)}.
\]
The sign is given by the product of the following four contributions:
the sign of \(\sigma\),
the Koszul sign given by permutation of the elements, as referenced above,
the Koszul sign arising from the permutation of \(\sigma x^{(1)}\) and \(m_{q_1,\dots,q_k}\),
and the explicit factor:
\[
(-1)^{(q_1+\cdots+q_r+1)(r_1+\cdots+r_k) + (l_2+r_2) + 2(l_3+r_3) + \dotsb + (k-1)(l_k+r_k)}
\]
expressed under the assumption that \(\sigma\) is a \((\bar l, \bar q, \bar r, \bar p)\)-shuffle.

%Notice that the Koszul sign associated to the action of \(\sigma\) on \(x\) is already incorporated in the notation \(\sigma x^{(1)} \ot \sigma x^{(2)} \ot \sigma x^{(3)}\).\Bruno{This is no obvious from the definition of $\sigma x$ given above (the sign is outside the braces).}

\subsubsection{Obstruction maps}

For each \(t \geqslant  0, \  k \geqslant  1, \  p_1, \dots, p_k \geqslant  1\) with \(t + k \geqslant  2\), the \defn{obstruction map} \(n^t_{p_1, \dots, p_k}\) is defined on a basis element
\[
\underbrace{(x_1 \ot\dotsb\ot x_{p_1})}_{w_1}
\ot \underbrace{(x_{P_1 + 1} \ot\dotsb\ot x_{P_1 + p_2})}_{w_2}
\ot\dotsb\ot
\underbrace{(x_{P_{k-1} + 1} \ot\dotsb\ot x_{P_k})}_{w_k},
\]
where \(P_i = p_1 + \dots + p_{i-1}\), as
\begin{equation}\label{eq:relation_maps_1}
	\sum_{\mathclap{\begin{array}{c}
			\scriptstyle 0 \leqslant s \leqslant t
	\end{array}}}
	\hspace{25pt}
	\sum_{\hspace*{10pt}\mathclap{\begin{array}{l}
				\scriptstyle I \,\sqcup J = \{1, \dots, k\} \\[-2pt]
				\scriptstyle I = \{i_1, \dots, i_a\} \neq \emptyset \\[-2pt]
				\scriptstyle J = \{j_1, \dots, j_b\}
	\end{array}}}
	\hspace*{35pt}  % space
	\sum_{\hspace*{0pt}\mathclap{\begin{array}{c}
				\scriptstyle 1 \leqslant q_1 \leqslant p_1 \\[-2pt]
				\scalebox{0.5}{\(\vdots\)} \\[-2pt]
				\scriptstyle 1 \leqslant q_a \leqslant p_a
	\end{array}}}
	\hspace*{3pt}	% space
	\pm \ m^{s}_{r,\,  p_{j_1}, \dots, p_{j_b}}
	\Big(m^{t-s}_{\frac{p_1, \dots, p_{i_a}}{q_1, \dots, q_a}}
	(w_{i_1} \!\ot\dotsb\ot w_{i_a}) \ot w_{j_1} \!\ot\dotsb\ot w_{j_b}\Big),
\end{equation}
where \(r = 1 + p_{i_1} + \dots + p_{i_a} - q_1 - \dots - q_a\)\,; minus the sum
\begin{equation}\label{eq:relation_maps_2}
	\sum_{1 \leqslant i \leqslant k}
	\hspace*{15pt}
	\sum_{\mathclap{1 \leqslant j \leqslant p_i-1}}
	\ (-1)^{p_1+\cdots+p_{i-1}+j-i} \ m^{t-1}_{p_1, \dots, j, p_i-j, \dots, p_k}
	\Big(w_1 \ot\dotsb\ot w_i^{(1)} \ot w_i^{(2)} \ot\dotsb\ot w_k\Big),
\end{equation}
where \(w_i^{(1)} = x_{P_i+1} \ot\dotsb\ot x_{P_i+j}\) and \(w_i^{(2)} = x_{P_i+j+1} \ot\dotsb\ot x_{P_{i+1}}\).

\medskip

The sign in \eqref{eq:relation_maps_1} is given by the product of the following three contributions:
the Koszul sign arising from the permutation of the elements,
the sign of the \((a,b)\)-shuffle corresponding to the partition \(I \sqcup J = \{1,\dots,k\}\)---computed by considering the blocks of sizes \(p_1-1,\dots,p_k-1\) and shuffling them according to the permutation determined by this partition (so, in particular, if all \(p_i=1\), there is no sign)---and the explicit factor
\[
(-1)^{(q_1+\dotsb+q_a - a + 1)\,(p_{j_1}+\cdots+p_{j_b} - b) + r - 1}.
\]
%The sign in \eqref{eq:relation_maps_2} is
%\[
%(-1)^{p_1+\cdots+p_{i-1}+j-i}.
%\]

\subsubsection{\(\BV_{\!\infty}\)-algebras}

For more details on the notions of \(\BV_{\!\infty}\)-algebra, \(\Gerst_\infty\)-algebra, and shifted \(\rL_\infty\)-algebra we refer to \cite{GCTV12}.

\begin{theorem}[{\cite{GCTV12}}]\label{t:obstruction vanishing}
	Let \(\big(A, \set{m_{p_1, \dots, p_k}^t}\big)\) be a \(\cBV_{\!\infty}\)-algebra.
	Then:
	\begin{enumerate}
		\item \(\big(A, \set{m_{p_1, \dots, p_k}^t}\big)\) is a \(\BV_{\!\infty}\)-algebra if and only if all \(n^t_{p_1, \dots, p_k}\) vanish.
		\item \(\big(A, \set{m^0_{p_1, \dots, p_k}}\big)\) is a \(\Gerst_\infty\)-algebra if and only if every \(n^0_{p_1, \dots, p_k}\) vanishes.
		\item \(\big(A, m^0_1, m^0_{1,1}, m^0_{1,1,1}, \dots\big)\) is a shifted \(\rL_\infty\)-algebra if and only if every \(n^0_{1, \dots, 1}\) vanishes.
	\end{enumerate}
\end{theorem}

\subsection{Weight filtration}

\subsubsection{\(\cBV_{\!(n)}\)-algebras}

We assign to each \(m^t_{p_1, \dots, p_k}\) in a set of generating maps the following \defn{weight}:
\[
\mathrm{weight}(m^t_{p_1, \dots, p_k}) \defeq t + p_1 + \dots + p_k - 1.
\]

\begin{definition}
	A \defn{\(\cBV_{\!(n)}\)-algebra} is a \(\cBV_{\!\infty}\)-algebra in which all generating maps of weight greater than \(n\) vanish.
\end{definition}

%\Bruno{we could say that this convention is coherent with the (correct) convention for $C_{(n)}$-algebras: ``The generating operations $\{m^0_p\}_{p\leqslant n-1}$ of a \(\cBV_{\!(n)}\)-algebra form a $C_{(n)}$-algebra. Anibal: I think that is implicit in the next sentence.}

\noindent As in \cref{t:extension from C_infty}, every \(\rC_{(n)}\)-algebra extends to a \(\cBV_{\!(n)}\)-algebra, and, similarly to \cref{t:obstruction vanishing}, a \(\cBV_{\!(n)}\)-algebra descends to a \(\BV_{(n)}\)-algebra, a \(\Gerst_{(n)}\)-algebra, or a shifted \(\rL_{(n)}\)-algebra if and only if the same sets of obstruction maps vanish.
Among these obstructions, only finitely many can be non-zero, namely those whose weight
\[
\mathrm{weight}(n^t_{p_1, \dots, p_k}) \defeq t + p_1 + \dots + p_k - 1
\]
is at most \(n\).

\subsubsection{Initial stages}\label{ss:explicit_descriptions}

We provide an explicit description of the generating and obstruction maps in the initial stages of the weight filtration.

\smallskip\noindent\textsc{Generators}\nopagebreak

\smallskip\noindent\textit{Weight 0}.
The only generating map of weight \(0\) is \(m^0_1\) of degree \(\bars{m^0_1} = 1\).

\smallskip\noindent\textit{Weight 1}.
The set of generating maps of weight \(1\) is:
\[
m^1_1 \colon A \to A, \qquad
m^0_2 \colon A^{\ot 2} \to A, \qquad
m^0_{1,1} \colon A \ot A \to A,
\]
of degrees
\[
\bars{m^1_1} = -1, \quad
\bars{m^0_2} = 0, \quad
\bars{m^0_{1,1}} = -1.
\]
They are required to have the following symmetric properties:
\[
m^0_{2} = m^0_{2} \circ (12), \qquad
m^0_{1,1} = m^0_{1,1} \circ (12),
\]
or, equivalently, for homogeneous elements \(a,b \in A\),
\[
m^0_{2}(a,b) = (-1)^{|a||b|} m^0_{2}(b,a), \qquad
m^0_{1,1}(a,b) = (-1)^{|a||b|} m^0_{1,1}(b,a).
\]

\smallskip\noindent\textit{Weight 2}.
The subset of independent generating maps of weight \(2\) is:
\begin{align*}
	&m^0_{3}      \colon A^{\ot 3} \to A,
	&& m^0_{1,2}    \colon A \ot A^{\ot 2} \to A,
	&& m^0_{1,1,1}  \colon A \ot A \ot A \to A, \\
	&m^1_{1,1}    \colon A \ot A \to A,
	&& m^1_{2}      \colon A^{\ot 2} \to A,
	&& m^2_{1}      \colon A \to A,
\end{align*}
whose degrees are:
\begin{align*}
	&\bars{m^0_{3}} = -1,   && \bars{m^0_{1,2}} = -2,   && \bars{m^0_{1,1,1}} = -3, \\
	&\bars{m^1_{1,1}} = -3, && \bars{m^1_{2}}   = -2,   && \bars{m^2_{1}}     = -3.
\end{align*}
They are required to have the following symmetry properties:
\begin{equation*}\label{eq:symmetry_weight2}
	\begin{split}
		m^0_3 \circ \big(\id - (12) + (132)\big) &= 0, \qquad \Big(m^0_3 \circ \big(\id - (23) + (123)\big) = 0,\Big) \\
		m^0_{1,2} \circ \big(\id - (23)\big) &= 0, \qquad \Big(m^0_{2,1} - m^0_{1,2} \circ (123) = 0,\Big) \\
		m^0_{1,1,1} \circ \big(\id - (12)\big) &= 0, \qquad m^0_{1,1,1} \circ \big(\id - (23)\big) = 0, \\
		m^1_{1,1} \circ \big(\id - (12)\big) &= 0, \qquad m^1_{2} \circ \big(\id - (12)\big) = 0.
	\end{split}
\end{equation*}

\smallskip\noindent\textit{Weight 3}.
There are 11 independent generators of weight 3:
\begin{align*}
	&m^0_{4}
	&& m^0_{1,3}
	&& m^0_{2,2}
	&& m^0_{1,1,2}
	&& m^0_{1,1,1,1}
	\\
	&m^1_{3}
	&& m^1_{1,2}
	&& m^1_{1,1,1}
	\\
	&m^2_{2}
	&& m^2_{1,1}
	\\
	&m^3_{1}
\end{align*}

\Bruno{Bottom Page 9: more terms can be simply written as brackets like \([m^1_1, m^0_{1,1}]\) in \(n^1_{1,1}\) or \([m^1_1, m^0_2]\) in \(n^1_2\). \\ Anibal: I think that would be good, but I do not have the confidence to make the changes.}

\medskip\noindent\textsc{Obstructions}\nopagebreak

\medskip\noindent\textit{Weight 1}.\par\nopagebreak
\medskip\noindent\renewcommand{\arraystretch}{1.25}
\begin{tabularx}{\textwidth}{|@{}r@{\hspace{2pt}}X@{}|}
	\hline
	% commutativity of differential and codifferential
	\(n^1_1\) & \( = \displaystyle m^0_1 \circ m^1_1 + m^1_1 \circ m^0_1 \)
	\( = [m^0_1, m^1_1] \) \\

	\hline
	% derivation of the bracket
	\,\(n^0_{1,1}\) & \( = \displaystyle
	m^0_1 \circ m^0_{1,1}
	- m^0_{1,1} \circ_1 m^0_1
	- m^0_{1,1} \circ_2 m^0_1
	= [m^0_1, m^0_{1,1}] \) \\

	\hline
\end{tabularx}

\medskip\noindent\textit{Weight 2}.\par\nopagebreak
\medskip\noindent\renewcommand{\arraystretch}{1.25}
\begin{tabularx}{\textwidth}{|@{}r@{\hspace{2pt}}X@{}|}
	\hline
	% leibniz
	\(n^0_{1,2}\) & \( = [m^0_1, m^0_{1,2}]
	- m^0_{2} \circ_1 m^0_{1,1}
	- (m^0_{2} \circ_2 m^0_{1,1}) \circ(12)
	+ m^0_{1,1} \circ_2 m^0_2 \) \\

	\hline
	% jacobi
	\,\(n^0_{1,1,1}\) & \( =
	[m^0_1, m^0_{1,1,1}] - \big(m^0_{1,1} \circ_1 m^0_{1,1}\big) \circ \big(\id + (123) + (132)\big) \) \\

	\hline
	% bracket is a coderivation
	\(n^1_{1,1}\) & \( =
	[m^0_1, m^1_{1,1}]
	- m^1_1 \circ m^0_{1,1}
	+ m^0_{1,1} \circ_1 m^1_1
	+ m^0_{1,1} \circ_2 m^1_1 \) \\

	\hline
	% failure of product being a coderivation is the bracket
	\(n^1_2\) & \( =
	[m^0_1, m^1_2]
	+ m^1_1 \circ_1 m^0_2
	- m^0_2 \circ_1 m^1_1
	- m^0_2 \circ_2 m^1_1
	- m^0_{1,1} \) \\

	\hline
	% codifferential
	\(n^2_1\) & \( =
	[m^0_1, m^2_1]
	+ m^1_1 \circ m^1_1 \) \\
	\hline
\end{tabularx}

\medskip\noindent\textit{Weight 3}.\par\nopagebreak
\medskip\noindent\input{sec/obstructions_weight3}

\medskip\noindent We describe explicitly the properties obstructed by the terms of weight at most 2:
%\medskip\noindent\textit{Obstructs}\par\nopagebreak

\medskip\noindent
\begin{tabular}{l l}
	\(n^1_1\) & \(m^0_1\) commuting with the operator \(m^1_1\). \\[2pt]
	\(n^0_{1,1}\) & \(m^0_1\) acting as a derivation of the shifted bracket \(m^0_{1,1}\). \\[2pt]
	\(n^0_{1,2}\) & \(m^0_2\) and \(m^0_{1,1}\) satisfying the Leibniz relation up to the homotopy \(m^0_{1,2}\). \\[2pt]
	\(n^0_{1,1,1}\) & \(m^0_{1,1,1}\) serving as a homotopy for the (shifted) Jacobi relation of \(m^0_{1,1}\). \\[2pt]
	\(n^1_{1,1}\) & \(m^1_{1,1}\) serving as a homotopy for the commutation relation between \(m^1_1\) and \(m^0_{1,1}\). \\[2pt]
	\(n^1_2\) & \(m^1_2\) serving as a homotopy for the relation expressing \(m^0_{1,1}\) as the commutator of \(m^1_1\) with \(m^0_2\). \\[2pt]
	\(n^2_1\) & \(m^2_1\) serving as a homotopy for the square-zero relation of \(m^1_1\). \\
\end{tabular}

%\begin{itemize}[leftmargin=18pt, label*=\(\diamond\)]
%	\item \(n^1_1\): \(m^0_1\) commuting with the operator \(m^1_1\).
%	\item \(n^0_{1,1}\): \(m^0_1\) acting as a derivation of the shifted bracket \(m^0_{1,1}\).
%	\item \(n^0_{1,2}\): \(m^0_2\) and \(m^0_{1,1}\) satisfying the Leibniz relation up to the homotopy \(m^0_{1,2}\).
%	\item \(n^0_{1,1,1}\): \(m^0_{1,1,1}\) serving as a homotopy for the (shifted) Jacobi relation of \(m^0_{1,1}\).
%	\item \(n^1_{1,1}\): \(m^1_{1,1}\) serving as a homotopy for the commutation relation between \(m^1_1\) and \(m^0_{1,1}\).
%	\item \(n^1_2\): \(m^1_2\) serving as a homotopy for the relation expressing \(m^0_{1,1}\) as the commutator of \(m^1_1\) with \(m^0_2\).
%	\item \(n^2_1\): \(m^2_1\) serving as a homotopy for the square-zero relation of \(m^1_1\).
%\end{itemize}\anibal{Convert to table. maybe obtruction | to}

\subsection{Yang--Mills kinematic algebra}

We now review an example of a \(\cBV_{\!(2)}\)-algebra arising as the kinematic algebra of Yang--Mills theory.
It was constructed and developed in \cite{Bonezzi2022GaugeStructureDoubleField, Bonezzi2023GaugeInvariantDoubleCopyQuartic, Bonezzi2024WeaklyConstrainedDoubleField}.

\medskip\noindent
Let \(\cO\) denote the vector space of smooth real-valued function on \(\R^d\) with the Minkowski metric \(\eta  =  \operatorname{diag}(+,-,\dots,-)\).
We use the following short hand notation throughout:
\[
\partial_\mu \lambda = \frac{\partial \lambda}{\partial t} + \sum_{i = 1}^{d-1} \frac{\partial \lambda}{\partial x^i} \,,
\qquad
\partial^\mu \lambda = \eta^{\mu \nu} \partial_\nu \lambda = \frac{\partial \lambda}{\partial t} - \sum_{i = 1}^{d-1} \frac{\partial \lambda}{\partial x^i} \,.
\]
Let \(\cK  =  \cZ \ot \cO\) where \(\cZ\) be the following graded vector space with a given basis:
\[
\begin{tikzcd}[row sep = 0]
	\cZ_0 & \cZ_1 & \cZ_2 & \cZ_3 \\
	\R\set{\theta_+} & \R\set{\theta_0, \dots, \theta_{d-1}} & \R\set{\theta_-} \\
	& \oplus & \oplus & \\
	& \R\set{s\theta_+} & \R\set{s\theta_0, \dots, s\theta_{d-1}} & \R\set{s\theta_-}. \\
\end{tikzcd}
\]
We will define multilinear maps on \(\cK\) using this basis, specifically, assigning to each basis element of \(\cZ^{\ot r}\) a sum of differential operator \(\cO^{\ot r} \to \cO\) parameterized by \(\cZ\).
More precisely, we will use the natural inclusion
\[
\Hom(\cZ^{\ot r}, \cZ \ot \Hom(\cO^{\ot r}, \cO)) \to \Hom(\cK^{\ot r}, \cK).
\]
Additionally, we will omit operators that are identically \(0\) and leave implicit the operator \(\id \ot\dotsb\ot \id\), which agrees with the product of functions.

\medskip\noindent The map \(m_1\) is defined by:
\begin{align*}
	&\theta_+ \mapsto \theta_\mu \partial^\mu + s\theta_+ \partial^\mu \partial_\mu,
	&&s\theta_+ \mapsto - s\theta_\mu \partial^\mu - \theta_-, \\
	&\theta_\mu \mapsto s\theta_\mu \partial^\mu \partial_\mu + \theta_- \partial_\mu,
	&&s\theta_\mu \mapsto - s\theta_- \partial_\mu, \\
	&\theta_- \mapsto s\theta_- \partial^\mu \partial_\mu,
	&&s\theta_- \mapsto 0.
\end{align*}
The map \(m_2\), which is assumed to satisfy \(m_2 \circ (12) = m_2\), is defined by:

\noindent\hspace*{5pt}
\begin{minipage}{0.2\textwidth}
	\begin{align*}
		&\theta_+ \ot \theta_+
		\mapsto \theta_+ , \\
		%%%
		&\theta_+ \ot \theta_\mu
		\mapsto \theta_\mu + s\theta_+( \partial_\mu \ot \id + \id \ot \partial_\mu), \\
		%%%
		&\theta_+ \ot \theta_-
		\mapsto -s\theta_\mu (\id \ot \partial^\mu), \\
		%%%
		&\theta_+ \ot s\theta_\mu
		\mapsto s\theta_\mu , \\
		%%%
		&\theta_+ \ot s\theta_-
		\mapsto s\theta_- ,
	\end{align*}
\end{minipage}
\begin{minipage}{0.5\textwidth}
	\begin{align*}
		&\theta_\mu \ot \theta_\nu
		\mapsto
		s\theta_\nu ( \partial_\mu \ot \id + 2\, \id \ot \partial_\mu ) \\
		&\phantom{(\theta_\mu, \theta_\nu) \mapsto}
		- s\theta_\mu ( \id \ot \partial_\nu + 2\, \partial_\nu \ot \id ) \\
		&\phantom{(\theta_\mu, \theta_\nu) \mapsto}
		+ s\theta_\rho \eta_{\mu\nu} ( \partial^\rho \ot \id - \id \ot \partial^\rho ), \\
		%%%
		&\theta_\mu \ot \theta_-
		\mapsto s\theta_- (\id \ot \partial_\mu), \\
		%%%
		&\theta_\mu \ot s\theta_\nu
		\mapsto -s\theta_- \eta_{\mu \nu}.
	\end{align*}
\end{minipage}

\medskip\noindent
The map \(m_3\) is defined by
\[
\theta_\mu \ot \theta_\nu \ot \theta_\rho
\mapsto
s\theta_\mu \eta_{\nu\rho} - 2s\theta_\nu \eta_{\mu\rho} + s\theta_\rho \eta_{\nu\nu}.
\]

In \cite{Bonezzi2022GaugeStructureDoubleField}, the authors verify that this maps define a \(\rC_{(2)}\)-algebra structure on \(\cK\).
By \cref{t:extension from C_infty}, this structure can be extended to a \(\cBV_\infty\)-algebra structure by choosing, for each \(	t \geqslant  0, \  k \geqslant  1, \  p_1, \dots, p_k \geqslant  1\) with \(t+k \geqslant 2\), a generating maps \(m^t_{p_1,\dots,p_k}\).
We now review the choices made by the authors in \cite{Bonezzi2023GaugeInvariantDoubleCopyQuartic, Bonezzi2024WeaklyConstrainedDoubleField}.

\medskip\noindent\textsc{Generators}\nopagebreak

\medskip\noindent\textit{Weight 0}.
We set \(m^0_1 \defeq m_1\).

\medskip\noindent\textit{Weight 1}.
We set \(m^0_2 \defeq m_2\).
The generating map \(m^1_1\) is defined by
\begin{align*}
	s\theta_+ \mapsto \theta_+, \quad
	s\theta_\mu \mapsto \theta_\mu, \quad
	s\theta_- \mapsto \theta_-.
\end{align*}
The final generating map of weight \(1\) is \(m^0_{1,1} \defeq [m^1_1, m^0_2]\).

\medskip\noindent\textit{Weight 2}.
We set \(m^0_3 \defeq m_3\).
The following generating maps are chosen to be \(0\):
\begin{align*}
	m^1_2 &= 0,	& m^2_1 &= 0,	& m^1_{1,1} &= 0.
\end{align*}
The authors explicitly define a map \(\theta_3 \colon \cK^{\ot 3} \to \cK\), found in \cite[Appendix~A]{Bonezzi2024WeaklyConstrainedDoubleField}, which we use to define \(m^0_{1,2} \defeq \theta_3 \circ (13)\).
The final generating map of weight 2 is \(m^0_{1,1,1} \defeq -[m^1_1, m^0_{1,2}]\).

\medskip\noindent\textit{Weight \(\geqslant\)\,3}. All other generating maps are chosen to be identically 0.
Therefore, the resulting structure on \(\cK\) is that of a \(\cBV_{\!(2)}\)-algebra.

\medskip\noindent\textsc{Obstructions}\nopagebreak

\medskip\noindent
We use the general list in \cref{ss:explicit_descriptions} to make explicit some of the obstruction maps arising from the above choices of generating maps.

\medskip\noindent\textit{Weight 1}.
The obstruction map \(n^1_1 = [m^0_1, m^1_1]\) is the d'Alambertian operator \(\square\) and \(n^0_{1,1} = [\square, m^0_{1,1}]\).

\medskip\noindent\textit{Weight 2}.
Three obstruction maps are identically 0:
\begin{align*}
	n^1_2 &= 0,	& n^2_1 &= 0,	& n^1_{1,1} &= 0.
\end{align*}
The other two, \(n^0_{1,2}\) and \(n^0_{1,1,1}\), are obstructions to the homotopy Leibniz and Jacobi relations respectively.
They are left uncomputed.

\medskip\noindent\textit{Weight 3}.
There are four obstruction maps that are identically 0:
\[
n^1_{1,2} = 0, \qquad n^2_{1,1} = 0, \qquad n^2_2 = 0, \qquad n^3_1 = 0.
\]
Two others are simplified compared to their general definition:
\[
n^1_3 = - [m^1_1, m^0_3] + m^0_{1,2} + m^0_{2,1}
\quad\text{and}\quad
n^1_{1,1,1} = [m^1_1, m^0_{1,1,1}].
\]
The remaining four obstruction maps are left uncomputed.

\Bruno{I am completely confused by the notation with the index \(/\mu\). For mathematicians, \(\partial_i=\frac{\partial}{\partial x_i}\), but still. I understand \(\partial_\mu\) and \(\partial^\mu\) as symbols denoting sums of partial derivatives, ok. But then, what does the assignment \(\theta_\mu \mapsto s\theta_\mu\, \partial^\mu \partial_\mu + \theta_-\,\partial_\mu\) mean? I would understand something like \(\theta_i \mapsto s\theta_i\, \partial^\mu \partial_\mu + \theta_-\,\partial_\mu\). \\
Anibal: That is what they mean, except that it would be  \(\theta_i \mapsto s\theta_i\, \partial^\mu \partial_\mu + \theta_-\,\partial_i\) }

\Bruno{The arrow could represent the inclusion: \(\hookrightarrow\).\\
Anibal: What arrow?}

\Bruno{Top of page 11: what we are considering is not the tensor product \(\partial_\mu \otimes \id\) but the product after the tensor product. \\
Anibal: I made a comment that the tensor product of realvalued functions is their product.}

\Bruno{I do not know what the operators \(\eta^{\mu\nu}\) and \(\eta_{\mu\nu}\) are.\\
The metric. these are its entries: mu = nu implies eta = 1 or -1 depending on if mu > 1 or not.}

\Bruno{Do you wish to recall the definition of the map \(\theta_3\) from \([{\rm BCDJH}24]\)? \\
Anibal: Super long, but I included a reference.}

\Bruno{There is a minus sign missing in the definition of \(n^1_{1,1,1}\). \\
Anibal: I do not know exactly where, sorry.}
