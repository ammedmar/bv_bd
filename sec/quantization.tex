% !TEX root = ../bv_bd.tex

\section{BV Quantization}

\subsection{Classical \(\BV\) complexes}

A \defn{classical \(\BV\) complex} is a cochain complex equipped with a \(\Pois_0\)-algebra structure, that is, a dg commutative and associative product  \(m \colon A \otimes A \to A\) and a dg Lie bracket \(b \colon A \otimes A \to A\) of degree \(1\) satisfying the \textit{Leibniz relation}:
\[
b(a, m(x,y)) = m(b(a,x), y) + (-1)^{(|a| + 1 - n)|x|} m(x, b(a,y))
\]
for all homogeneous \(a, x, y \in A\).

\medskip\noindent The operad controlling these algebras is, after degree involution, the homology of the little disk operad.
Explicitly, the operad \(\Pois_0\) is presented by the following \textit{quadratic data}:

\medskip\noindent \(\diamond\) \(E_{\Pois_0} \defeq \KK\set{\m} \oplus \KK\set{\b}\) where
\[
\m \defeq \M, \quad |\m| = 0 \quad ; \quad \b \defeq \B, \quad |\b| = 1
\]
with a trivial symmetric action.

\medskip\noindent \(\diamond\) \(R_{\Pois_0}\) is spanned, as an \(\Sy\)-module, by the following 3 relations:

\begin{tabular}{ll}
	\textnormal{\textsc{associativity:}} & \(\LL{1}{2}{3} - \RR{1}{2}{3}\) \\
	\textnormal{\textsc{jacobi:}} & \(\BBB{1}{2}{3}+ \hspace*{-5pt} \BBB{2}{3}{1}+ \hspace*{-5pt} \BBB{3}{1}{2}\) \\
	\textnormal{\textsc{leibniz:}} & \(\LBM{1}{2}{3}- \hspace*{-5pt} \LMB{1}{3}{2}-\RMB{1}{2}{3}\)~.
\end{tabular}

\begin{example}
	Let \(M\) be a finite-dimensional smooth manifold.
	The graded vector space \(\PV^\bullet(M)\) of \defn{polyvector fields}, with vector fields in degree \(-1\),  carries a \(\Pois_0\)-algebra structure with the Schouten bracket, the extension of the Lie bracket of vector fields via the Leibniz relation.

	We can think of \(\PV^\bullet(M)\) as the ring of functions on the shifted cotangent bundle \(\rT^*[-1] M\).
	The \(\Pois_0\) structure on \(\rT^*[-1] M\) is a supersymetric generalization of the Poisson structure on the ring of functions on \(\rT^*M\).

	Choosing a function \(S \colon M \to \R\) satisfying the \textit{classical master equation} \(\{S,S\} = 0\) we can promote this \(\Pois_0\)-algebra to \(\dgVec\) with differential \(d = \{S, -\}\).
	Its underlying cochain complex is a model for the derived critical locus of \(S\) \cite[\S?]{CostelloGwilliam2021FactorizationAlgebrasQFTv2}.
\end{example}

\begin{example}
	For a bounded cochain complex \((V, d_V)\) consider the \textit{functions on its shifted cotangent bundle}
	\[
	\cO(\rT^*[-1] V) \defeq \big(\Sym(V^\vee \ot V[1]), d\big).
	\]
	We can define a \(\Pois_0\) structure on this complex by extending the evaluation map using the Leibniz relation.
\end{example}

We remark that the homology of the little disk operad, isomorphic to \(\rT\Pois_0\), controls so-called \textit{Gerstenhaber algebras}.
Therefore, the category of classical \(\BV\) complexes is isomorphic to the category of chain complexes with a Gerstenhaber algebra structure.

\subsection{BV algebras}

An algebra over the homology of the little framed disk operad is referred as a \defn{Batalin--Vilkovisky algebra} or \defn{\(\BV\) algebra}.
Explicitly, this is a \(\rT\Pois_0\)-algebra \((A, d, m, b)\) together with a linear map \(\Delta \colon A \to A\) of degree \(1\) satisfying:
\[
[\Delta, \Delta] = 0, \quad
[\Delta, d] = 0, \quad
[\Delta, m] = b, \quad
[\Delta, b] = 0.\footnote{This last identity follows from the others.}
\]

These are algebras over the operad \(\BV\) which we present using the following inhomogeneous quadratic data

....

\begin{example}
	Continuing with \cref{ex:polyvectors}.
	A choice of volume form \(\mu\) on \(M\) gives an isomorphism
	\[
	PV --> dR
	\]
	which we use to transfer the de Rham differential.
	The resulting operator \(\Delta_\mu \defeq \iota^{-1}_\mu \circ d_{dR} \circ \iota_\mu\) makes this \(\Pois_0\)-algebra into a \(\BV\)-algebra.
\end{example}

\subsection{Quantum \(\BV\) complexes}

A \defn{quantum \(\BV\) complex} is a flat graded \(\KK[[\hbar]]\)-module \(A^q\) with a \(\KK[[\hbar]]\)-linear differential \(d\), a graded commutative and associative product of degree \(0\), and degree \(1\) Lie bracket \(b\) satisfying both the Leibniz rule and
\[
[d, m] = \hbar \cdot b.
%d(ab) = (da)b + (-1)^{|a|} a(db) + \hbar\{a,b\}.
\]
%for all homogeneous \(a, b\in A\)
These are also referred to as \defn{Beilinson--Drinfeld algebras} or \defn{\(\BD\) algebras} after \cite{bibid}.

The main contribution of this note is to provide a new operadic description of these algebras.
Specifically, in \cref{label} we show that these are cochain complexes of \(\KK\)-vector spaces with the structure of an (absolute) algebra over the Koszul dual cooperad of \(\BV\).

%or \defn{\(\BD\)-algebra} \((A^q, d, m, b = \{-,-\})\) is a graded commutative algebra \((A^q, m)\), flat as a module over \(\KK[[\hbar]]\), equipped with a degree \(1\) Poisson bracket \(\{-,-\}\) and an \(\KK[[\hbar]]\)-linear differential \(d\), such that for all homogeneous \(a, b\in A\) one has
%\[
%d(ab) = (da)b + (-1)^{|a|} a(db) + \hbar\{a,b\}.
%\]

\medskip When working over the polynomial ring \(\KK[\hbar]\) instead of the ring of power series \(\KK[[\hbar]]\) we refer to the analogous structure as a \defn{polynomial \(\BD\)-algebra}.

\begin{example}
	Consider the \(\Pois_0\)-algebra on polyvector fields with the \(\BV\) operator \(\Delta_\mu\) defined by a volume form \(\mu\) as described in \cref{label}.
	We now consider a \textit{classical action} \(S \colon M \to \KK\) as in \cref{label}.
	We define a new volume form\anibal{volume form?}
	\[
	\mu_S \defeq e^{S/\hbar} \cdot \mu
	\]
	where \(\hbar\) is an unspecified parameter.
	The transfer of the de Rham differential through the isomorphism defined by \(\mu_S\) satisfies:
	\[
	\hbar\Delta_{\mu_S} =
	\set{S, -} + \hbar\Delta_\mu.
	\]
	Setting \(d \defeq \set{S, -} + \hbar\Delta_\mu\) we obtain a polynomial \(\BD\)-algebra on \(\PV^\bullet(M) \ot \KK[\hbar]\).


	For any \(f \in C^\infty(M)\),
	\[
	[f]_{\Delta_{\mu_S}} = \angles{f}_{\mu_S} \cdot [1]_{\Delta_{\mu_S}}.
	\]
	where \([\,-\,]_{\Delta_{\mu_S}}\) denotes the class in \(\Delta_{\mu_S}\) homology and
	\[
	\angles{f}_{\mu_S} \defeq
	\frac{\int f \cdot \mu_S}{\int \mu_S}.
	\]
	This last integral is interpreted as the \textit{expectation value} of the \textit{observable} \(f\).
\end{example}

Likewise, we can restrict to \(\hbar\neq 0\) by setting
\[
A_{\hbar\neq 0}\defeq A^q\otimes_{\KK[[\hbar]]}\KK((\hbar)).
\]
In this case we obtain only a cochain complex, and its cohomology need not inherit an algebra structure, unlike \(H^*(A_{\hbar=0})\).


\noindent-------

\medskip\noindent Given a BD-algebra \(A^q\), we can restrict to \(\hbar = 0\) by setting
\[
A^q_{\hbar=0} \defeq A \otimes_{\KK[[\hbar]]}\KK[[\hbar]]/(\hbar).
\]
The induced differential on \(A_{\hbar = 0}\) is a derivation, hence \(A^q_{\hbar = 0}\) is a \(\Pois_0\)-algebra.

\medskip\noindent A \defn{\(\BV\)-quantisation} of a \(P_0\)-algebra \(A\) is a BD algebra \(A^q\) such that \(A^q_{\hbar=0} = A\).



